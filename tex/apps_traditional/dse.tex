As we saw in Chapter~\ref{chap:mapping}, a central step of model-based software synthesis is a \ac{DSE} step for finding mappings, among others.
We know the mapping space is intractably large and complex and we cannot find the actual optima in the space for any real-life problem sizes.
The best we can hope for are near-optimal mappings in a reasonable amount of time.
Thus, we focus both on the quality of the mappings as well as the time required.
This section will focus on \ac{DSE} for finding near-optimal mappings, as defined in Section~\ref{sec:mapping_problem}. 
We will see many applications of the structures defined and analyzed in Chapter~\ref{chap:mapping_structures}

\subsection{Heuristics and Metaheuristics}

Generally in \ac{DSE} we distinguish between two approaches for dealing with these kinds of intractable problems, heuristics and meta-heuristics (cf. Section~\ref{sec:mapping_problem}).
Recall that mapping heuristics are domain-specific algorithms that exploit the specific domain-knowledge to find a solution based on a pre-defined model of the problem, whereas meta-heuristics rely on an iterative evaluation of the points.
As we outlined above, different heuristics and meta-heuristics come with trade-offs between the exploration time required to find a solution and the quality of said solution.
This is certainly the case for many discrete optimization problems in general, the mapping problem being no exception~\cite{goens_mcsoc16}.
Commonly, meta-heuristics tend to find better results provided enough time, but require accordingly more time to do so.

A particular difficulty of comparing mapping approaches and algorithms are the different models used by different algorithms~\cite{goens_mcsoc16}.
With \mocasin we designed a common framework that allows us to compare between mapping algorithms~\cite{menard_rapido21}.
In particular, in \mocasin we have two heuristics for mapping: the \ac{GBM} heuristic~\cite{castrillon_dac12} and a static mapping variant~\cite{menard_rapido21} of the \ac{CFS} scheduler from Linux.
Additionally, we have implemented genetic algorithms based on and inspired by those found in Sesame~\cite{erbas2006multiobjective,quan2014towards,goens_mcsoc16}, a simulated annealing~\cite{simulated_annealing} mapping algorithm and a tabu search~\cite{tabu_search}.
We also have a simple random walk algorithm for reference.
A survey of these mapping algorithms, among others can be found in~\cite{singh2013mapping}.
We implemented these algorithms for \mocasin and this thesis to have a basis for comparison from established literature.

We first compare these mapping algorithms to establish a baseline. 
We execute a random walk $500$ random iterations.
For the genetic algorithm we run an evolutionary $\mu + \lambda$ strategy for $20$ generations of population size $10$, crossover rate of $1$ with probability $0.35$ and mutation probability $0.5$, with a tournament selection with tournament size $4$.
For the \ac{GBM} algorithm we set the parameters as \texttt{bx\_m} of $1$, \texttt{bx\_p} of $0.95$, \texttt{by\_m} of $0.5$,\texttt{by\_p} of $0.75$, 
The simulated annealing heuristic we execute with an initial temperature of $1$ and a final temperature of $0.1$, with a temperature proportionality constant of $0.5$  and a random movement starting radius of $5$.
Finally, for the tabu search mapper we set a maximum of $30$ iterations, each of size $5$ and with a move set size of $10$ and tabu tenure of $5$, and a random candidate move update radius of $2$.
These parameters were chosen such that the mappers seemed to yield good result, yet not systematically (e.g. using something like Bayesian optimization or general (hyper-)parameter optimization approaches).
A deliberate choice in the parameters however is that the exploration times should be comparable between the meta-heuristics, i.e. such that the iterative mappers evaluate a similar amount of mappings.

\begin{figure}[h]
	\centering
   \resizebox{0.95\textwidth}{!}{\input{generated/heuristics_vs_metaheuristics.tex}}
	\caption{Comparison of multiple mapping heuristics and metaheuristics on the \ac{E3S} benchmarks.}
	\label{fig:metric_comparison}
\end{figure}

Figure~\ref{fig:metric_comparison} shows a comparison of the different heuristics and metaheuristics on the \ac{E3S} benchmarks.
Each of the metaheuristics that require random data we execute $10$ times and show the variation as calculated by the unbiased estimator of the standard deviation of the multiple sampled times.
The execution times vary obviously depending on the different benchmark applications and on the platforms they run on.
The absolute values of these times, however, are not interesting for comparing the mapping algorithms.
We thus norm the values of the execution times, taking the results of the genetic algorithms as baseline.
We then aggregate all values with the geometric mean.
The error bars in the plot are calculated by taking the average value $\plusminus$ the estimated standard deviation and norming each of the two extremes, the results of which are the extremes shown in the plot.

We first examine the results for the Odroid XU4 architecture.
The two heuristics find considerably lower results in average, but they do so in considerably less time.
More concretely, they yield about an order of magnitude worse results in about an order of magnitude less time.
The results of the random walk heuristic are significantly worse than those of the more structured metaheuristics, even though it takes a comparable amount of time.
This is due to a deliberate choice, since as explained above, the number of random mappings was chosen specifically to be comparable to the number the number of mappings evaluated by the other meta-heuristics.
Since $500$ mappings is not a small amount, it is not terribly surprising that the random mapper beats the two heuristics.
Finally, the best mappings are found by the simulated annealing meta-heuristic, albeit only by $3\%$ compared to the genetic algorithm.

When we turn our attention to the significantly larger and more complex MPPA3 Coolidge architecture, we see that the picture changes drastically.
The marked difference between heuristics and meta-heuristics disappears in this case.
The \ac{GBM} heuristic is on par with the random walk results in average, while taking substantially less time.
This is simply explained by the significantly larger design space of mapping to the MPPA3 Coolidge.
In this case, the genetic algorithm significantly outperforms both other meta-heuristics, by a factor of $4-5$, while taking less time.
The most striking result here, however, is the extremely good performance of the static CFS heuristic.
This good performance is misleading at first, a perhaps more honest assessment of the results is that \emph{all other (meta-) heuristics perform poorly}.
We can interpret this as a consequence of the growing design space and its complexity, which affects the metaheuristics, while the static CFS mapper can still leverage domain-specific knowledge to find fairly good mappings.

TODO: add and test genetic with CFS initials.

\subsection{Leveraging Symmetries}
Talk about~\cite{goens_taco17} + new stuff (mpsym)

\subsection{Leveraging Metric Spaces}
Improving DSE algorithms with metric spaces?

Talk about gradient descent: why it does not work w/o metric space structure, and why it should with it.
\begin{figure}[h]
	\centering
   \resizebox{0.95\textwidth}{!}{\input{generated/geometric_heuristics_coolidge.tex}}
	\caption{The effect of embedding-based representations in metaheuristics that leverage the geometry on the MPPA3 Coolidge platform.}
	\label{fig:coolidge_geometric}
\end{figure}

\subsection{All Representations}
\begin{figure}[h]
	\centering
   \resizebox{0.95\textwidth}{!}{\input{generated/multiple_representations_exynos.tex}}
	\caption{Comparison of the effects of multiple representations on the Odroid XU4 platform.}
	\label{fig:multiple_representations_exynos}
\end{figure}

\begin{figure}[h]
	\centering
   \resizebox{0.95\textwidth}{!}{\input{generated/multiple_representations_coolidge.tex}}
	\caption{Comparison of the effects of multiple representations on the MPPA3 Coolidge platform.}
	\label{fig:multiple_representations_exynos}
\end{figure}