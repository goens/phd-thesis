Depending on the context, different representations of the mapping space might offer extremely efficient ways of answering a particular type of question about a mapping.
In the \texttt{SimpleVector} representation, the question ``Is Task$_A$ mapped to PE$_3$'' is very easy to answer, whereas answering the question ``is the expected latency between Task$_B$ and Task$_C$ below $10$ ns?'' is more difficult to answer.
Conversely, in a representation based on a metric space topology defined by the communication distances between hardware resources, the difficulty of these two last questions might be reversed.
In order to efficiently find mappings thus, depending on the objectives, an algorithm might want to use a representation or the other, or perhaps a combination of them.

We aim to do so by defining a language that shall permit to query and combine ontologies in a uniform fashion.
This language will be based in propositional or first-order logic, and also include implicit domains and ontology relations to define and combine statements in the different ontological representations of mappings.

There is a distinct advantage in defining a language, as opposed to simply defining a series of programming interfaces to the different representations, and letting algorithm programmers combine them in a programmatic way using a general-purpose language, like Python.
By defining a domain-specific query language, we are creating a new level of abstraction that will hopefully allow researchers to reason about the mapping problem in new ways, transcending the simple usages to combine queries that will be presented in this section.

We describe some representations of mappings (ontologies) in a superficial way, focusing on the kinds of questions they are well-suited to answer.
We then proceed to give examples of questions that might be asked, and how these could be combined using the language.
Finally, we describe a draft of the language and a query-evaluation strategy that might be usable to find mappings using this language.

\subsection{Example Ontologies}

In the following we present some ontologies as examples of where questions could be asked, describing only the rough idea and what kinds of questions might be easy to answer in them.

\paragraph{\texttt{SimpleVector} Representation}
This representation is, as described in the problem formulation, the typical mapping representation that uses vectors of the form $ m = (p_1,\ldots,p_k,c_1,\ldots,c_l)$ to define the mapping.
This representation is well-suited to answer questions of the form:
\begin{itemize}
  \item Is task $A$ mapped to $PE_1$?
  \item Does $PE_2$ execute any process?
 \item Do tasks $A$ and $B$ execute on the same PE?
\end{itemize}

\paragraph{\texttt{Symmetries} Representation}
This representation normalizes mappings that are equivalent to a single (canonical) mapping, while still using the vector form.
This representation is well-suited to answer questions of the form:
\begin{itemize}
\item Is this mapping equivalent to mapping $m'$?
\item Do tasks $A$ and $B$ execute on the same PE?
\end{itemize}

\paragraph{Metric Space Representation without Embeddings}
This representation uses the communication topology to define meaningful distances between PEs, and by extension, between mappings.
This representation is well-suited to answer questions of the form:
\begin{itemize}
\item Is this mapping very similar to mapping $m'$? \textbf{(can give false positives)}
\item Is the expected latency between tasks $A$ and $B$ under $10$ ns?
\item Do tasks $A$ and $B$ execute on the same PE?
\end{itemize}

\paragraph{\texttt{MetricSpaceEmbedding} Representation}
This is similar to the metric space representation, but computationally much more efficient, at the expense of a bit of accuracy.
This representation is well-suited to answer questions of the form:
\begin{itemize}
\item Is this mapping very similar to mapping $m'$? \textbf{(can give false positives)}
\item Is the expected latency between tasks $A$ and $B$ under $10$ ns?
\end{itemize}

\paragraph{Metric Space Symmetry Representation}
This representation combines the metric space structure with the symmetry reduction. It is computationally very inefficient (WIP to improve!)
\begin{itemize}
\item Is this mapping very similar to mapping $m'$? 
\item Is this mapping equivalent to mapping $m'$?
\item Is the expected latency between tasks $A$ and $B$ under $10$ ns?
\item Do tasks $A$ and $B$ execute on the same PE?
\end{itemize}

\paragraph{Inclusion-Hierarchy-Based Spatial Ontology}
This representation is not implemented, and thus, not very mature.
It should have a way to define a PE hierarchy with refinements (PEs $\in$ clusters $\in$ chips) and similarly for hierarchical applications (like Composites in Ptolemy II).

\subsubsection{The Language}
It should be based on propositional logic or first-order logic (with quantifiers only valid for some domains).
It would have the following implicit domains:
\begin{itemize}
\item Mappings
\item Processing elements
\item Hardware communication resources
\item Tasks (or Processes or Actors or Composites, etc.)
\item Communication channels
\item Distances (e.g. the natural numbers)
\item Hierachical strutures? E.g. composites in the application or clusters in a multicore (This might be difficult to get right, in an instance-independent way. Perhaps only having abstract hierarchy levels?)
\end{itemize}

The statements would always refer to a mapping, i.e. every mapping in the mapping space would either satisfy or not such a statement, and our solver would try to find a solution to a statement (i.e. a mapping) or a set of such solutions.

The following are examples of statements that might be made of a mapping.
\begin{enumerate}
\item Is task $A$ mapped to $PE_1$?
\item Does $PE_2$ execute any process?
\item Do tasks $A$ and $B$ execute on the same PE?
\item Is this mapping equivalent to mapping $m'$?
\item Is this mapping very similar (distance < 10) to mapping $m'$? 
\item Is the expected latency between tasks $A$ and $B$ under $10$ ns?
\item Do tasks $A$ and $B$ execute on the same PE?
\item Are all task-groups at the highest level of hierarchy in the application mapped to PE-groups at the highest level of the hierachy? (e.g. composites at the highest level to the same PE cluster)
\end{enumerate}

Of course, the idea is to be able to combine them, for example:
`` Is this mapping very similar to $m'$ (distance $\leq 100$) \textbf{and} \textbf{not} Is this mapping equivalent to $m'$ \textbf{and} (\textbf{there exists} a PE $p$ such that tasks $A, B$ and $C$ are mapped to $p$ \textbf{or} (the expected latency between tasks $A$ and $B$ is small than $10$ \textbf{and} the expected latency between taskts $B$ and $C$ is smaller than $10$ \textbf{and} the expected latency between tasks $A$ and $C$ is smaller than $15$))''
\subsection{Evaluation Strategies}
A central point of this implementation and project is an evaluation strategy of how to find a mapping in this space.
An exhaustive search is infeasible (otherwise we would not be doing this in the first place!).
However, the different ontologies have different methods for iterating through the mappings, which could be leveraged by the approach. A rough sketch of what we would want to do so far is the following:
\begin{itemize}
\item Reorder statements to have an ideal execution strategy (this might happen at runtime)
\item Have some sort of iterator for the mappings that depends on the ontology
\item Have ontology-specific constraints be available to pass to the iterators
\end{itemize}