Computer hardware keeps increasing in complexity.
Software design needs to keep up with this.
The right models and abstractions empower developers to leverage the novelties of modern hardware.
This thesis deals primarily with \aclp*{MoC}, as a basis for software design, in a family of methods called software synthesis.

We focus on \aclp*{KPN} and dataflow applications as abstractions, both for programming and for deriving an efficient execution on heterogeneous multicores.
The latter we accomplish by exploring the design space of possible mappings of computation and data to hardware resources.
Mapping algorithms are not at the center of this thesis, however.
Instead, we examine the mathematical structure of the mapping space, leveraging its inherent symmetries or geometric properties to improve mapping methods in general.

This thesis thoroughly explores the process of model-based design, aiming to go beyond the more established software synthesis on dataflow applications.
Starting at the problem of assessing these mehtods through benchmarking, we examine the general goals of benchmarks formally, as well as the role modern machine learning methods play in them.

We explore different established semantics, stretching the limits on \aclp*{KPN}.
We also discuss novel models, like Reactors, which are designed to be a deterministic, adaptive model with time as a first-class citizen.
By investigating abstractions and transformations in the Ohua language for implicit dataflow programming, we also focus on programmability.

The focus of the thesis is in the models and methods, but we evaluate them in diverse use-cases, generally centered around \aclp*{CPS}.
From the 5G telecommunication standard, through the automotive and signal processing domains.
We even go beyond embedded systems and discuss use-cases in \acs*{GPU} programming and microservice-based architectures.