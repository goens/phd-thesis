The \mocasin framework supports three input formats at the time of this writing: \texttt{tgff},\texttt{MAPS} and \texttt{sdf3}.
We will discuss the first two for benchmarking here, while the \texttt{sdf3} format will be discussed in Section~\ref{sec:level_graphs}.

\subsection{CPN Benchmarks}

The first input format for \mocasin is the \texttt{MAPS} format, which uses benchmarks written in the \ac{CPN} language (cf. Section~\ref{sec:kpn_basic}).
A \ac{CPN} application is compiled using the \ac{MAPS} flow, which evolved into the commercial tool suite from SLX.
SLX generously provided access to their tool suite for this benchmarking, as well as some \ac{CPN} benchmarks. 

The tool flow compiles an application using a \acp{pthread} back-end with instrumentation to record all tokens read and written in a trace.
Since this traces are deterministic and depend only on the inputs, not on the execution order, they can be used to replay and simulate the execution (cf. Section~\ref{sec:simulating_mappings}).
In an instrumented run, the \ac{MAPS} flow also executes each process in isolation (with the stored tokens), gathering information about the precise instructions executed.
This is used for performance estimation using an abstract processor model\cite{eusse2014pre}.
The performance estimation for each process, together with the execution traces can then be used to simulate a mapping with \mocasin, as explained in Section~\ref{sec:simulating_mappings}.

We use three \ac{CPN} benchmarks for evaluation.
The first benchmark is the audio filter example as seen in Chapter~\ref{chap:mapping}, which takes a stereo file and implements a low-pass filter after transforming to the frequency, and transforming it back afterwards.
This benchmark has $8$ processes and $15$ channels (some channels, used to send parameters, are not depicted in Figure~\ref{fig:audio_filter_graph}).
The second benchmark we consider is an embedded pedestrian-recognition application using an algorithm based on the \ac{HOG} technique\index{HOG}.
This benchmark was kindly provided by SLX and consists of $10$ processes with $33$ channels.
Finally, we use a speaker recognition application, as described in~\cite{bouraoui2019comparing}.
Figure~\ref{fig:speaker_recognition} shows the graph of the speaker recognition application. \index{speaker recognition}
The speaker recognition application has $12$ processes and $33$ channels.

\begin{figure}[h]
	\centering
\resizebox{0.9\textwidth}{!}{
     
\begin{tikzpicture}[>=latex,line join=bevel,]
\Large%
\node (process_source)  [draw,ellipse] {source};
  \node (process_readwave_stage1) [draw,ellipse,below left=of process_source] {readwave};
  \node (process_hamming_stage2)  [draw,ellipse,right =of process_readwave_stage1] {hamming};
  \node (process_FFT_stage3) [draw,ellipse,below =of process_hamming_stage2] {FFT};
  \node (process_melFreqWrap_stage4) [draw,ellipse,left =of process_FFT_stage3] {melFreqWrap};
  \node (process_DCT_stage5) [draw,ellipse,below =of process_melFreqWrap_stage4] {DCT};
  \node (process_ShifterDLP) [draw,ellipse,right =of process_DCT_stage5] {ShifterDLP};
  \node (process_Worker_0) [draw,ellipse,below right = of process_ShifterDLP] {Worker\_0};
  \node (process_Worker_1) [draw,ellipse,left = of process_Worker_0] {Worker\_1};
  \node (process_Worker_2) [draw,ellipse,left = of process_Worker_1] {Worker\_2};
  \node (process_Worker_3) [draw,ellipse,left = of process_Worker_2] {Worker\_3};
  \node (process_sink) [draw,ellipse,below = of process_Worker_1] {sink};
  \draw [->] (process_source) -- (process_readwave_stage1);
%  \definecolor{strokecol}{rgb}{0.0,0.0,0.0};                  _
%  \pgfsetstrokecolor{strokecol}
%  \draw (631.0bp,507.0bp) node {testFile};
  \draw [->,opacity=0.5,looseness=1] (process_source) -- (process_ShifterDLP);
%  \draw (745.5bp,354.0bp) node {speakers};
  \draw [->,opacity=0.5,looseness=1] (process_source) -- (process_ShifterDLP);
%  \draw (828.0bp,354.0bp) node {nbSpeakers};
  \draw [->,opacity=0.5,looseness=1] (process_source) -- (process_Worker_0);
%  \draw (126.0bp,315.0bp) node {nbSpeakers};
  \draw [->,opacity=0.5,looseness=1] (process_source) -- (process_Worker_1);
%  \draw (30.0bp,315.0bp) node {nbSpeakers};
  \draw [->,opacity=0.5,looseness=1] (process_source) -- (process_Worker_2);
%  \draw (970.0bp,315.0bp) node {nbSpeakers};
  \draw [->,opacity=0.5,looseness=1] (process_source) -- (process_Worker_3);
%  \draw (886.0bp,315.0bp) node {nbSpeakers};
  \draw [->,opacity=0.5,looseness=1] (process_readwave_stage1) -- (process_hamming_stage2);
%  \draw (477.0bp,431.0bp) node {tokenFrame};
  \draw [->,opacity=0.5,looseness=1] (process_readwave_stage1) -- (process_hamming_stage2);
%  \draw (399.5bp,431.0bp) node {nbFrames};
  \draw [->,opacity=0.5,looseness=1] (process_readwave_stage1) -- (process_FFT_stage3);
%  \draw (348.5bp,393.0bp) node {nbFrames};
  \draw [->,opacity=0.5,looseness=1] (process_readwave_stage1) -- (process_melFreqWrap_stage4);
%  \draw (504.0bp,354.0bp) node {sampleRate};
  \draw [->,opacity=0.5,looseness=1] (process_readwave_stage1) -- (process_melFreqWrap_stage4);
%  \draw (573.5bp,354.0bp) node {nbFrames};
  \draw [->,opacity=0.5,looseness=1] (process_readwave_stage1) -- (process_DCT_stage5);
%  \draw (310.5bp,315.0bp) node {nbFrames};
  \draw [->,opacity=0.5,looseness=1] (process_readwave_stage1) -- (process_Worker_0);
%  \draw (254.5bp,276.0bp) node {nbFrames};
  \draw [->,opacity=0.5,looseness=1] (process_readwave_stage1) -- (process_Worker_1);
%  \draw (180.5bp,276.0bp) node {nbFrames};
  \draw [->,opacity=0.5,looseness=1] (process_readwave_stage1) -- (process_Worker_2);
%  \draw (695.5bp,276.0bp) node {nbFrames};
  \draw [->,opacity=0.5,looseness=1] (process_readwave_stage1) -- (process_Worker_3);
%  \draw (619.5bp,276.0bp) node {nbFrames};
  \draw [->,opacity=0.5,looseness=1] (process_hamming_stage2) -- (process_FFT_stage3);
%  \draw (404.5bp,354.0bp) node {token\_H\_Frame};
  \draw [->,opacity=0.5,looseness=1] (process_FFT_stage3) -- (process_melFreqWrap_stage4);
%  \draw (411.5bp,276.0bp) node {spectrumVector};
  \draw [->,opacity=0.5,looseness=1] (process_melFreqWrap_stage4) -- (process_DCT_stage5);
%  \draw (476.0bp,198.0bp) node {logms};
  \draw [->,opacity=0.5,looseness=1] (process_melFreqWrap_stage4) -- (process_DCT_stage5);
%  \draw (405.0bp,198.0bp) node {codesize};
  \draw [->,opacity=0.5,looseness=1] (process_DCT_stage5) -- (process_Worker_0);
%  \draw (374.5bp,121.0bp) node {dct\_frame};
  \draw [->,opacity=0.5,looseness=1] (process_DCT_stage5) -- (process_Worker_1);
%  \draw (298.5bp,121.0bp) node {dct\_frame};
  \draw [->,opacity=0.5,looseness=1] (process_DCT_stage5) -- (process_Worker_2);
%  \draw (579.5bp,121.0bp) node {dct\_frame};
  \draw [->,opacity=0.5,looseness=1] (process_DCT_stage5) -- (process_Worker_3);
%  \draw (464.5bp,121.0bp) node {dct\_frame};
  \draw [->,opacity=0.5,looseness=1] (process_ShifterDLP) -- (process_Worker_0);
%  \draw (713.5bp,121.0bp) node {SPW0};
  \draw [->,opacity=0.5,looseness=1] (process_ShifterDLP) -- (process_Worker_1);
%  \draw (651.5bp,121.0bp) node {SPW1};
  \draw [->,opacity=0.5,looseness=1] (process_ShifterDLP) -- (process_Worker_2);
%  \draw (849.5bp,121.0bp) node {SPW2};
  \draw [->,opacity=0.5,looseness=1] (process_ShifterDLP) -- (process_Worker_3);
%  \draw (769.5bp,121.0bp) node {SPW3};
  \draw [->,opacity=0.5,looseness=1] (process_Worker_0) -- (process_sink);
%  \draw (447.5bp,47.0bp) node {str\_min\_0};
  \draw [->,opacity=0.5,looseness=1] (process_Worker_1) -- (process_sink);
%  \draw (348.5bp,47.0bp) node {str\_min\_1};
  \draw [->,opacity=0.5,looseness=1] (process_Worker_2) -- (process_sink);
%  \draw (713.5bp,47.0bp) node {str\_min\_2};
  \draw [->,opacity=0.5,looseness=1] (process_Worker_3) -- (process_sink);
%  \draw (527.5bp,47.0bp) node {str\_min\_3};
%
\end{tikzpicture}

 }
   \caption{The \ac{KPN} graph of the speaker recognition application.}
   \label{fig:speaker_recognition}
\end{figure}

\subsection{The E3S Benchmarks}

The second input of \mocasin we use for benchmarking is \texttt{tgff}.
The \texttt{tgff} format comes from \acf{TGFF}, a random task graph generator~\cite{dick1998tgff}. 
However, the same author also published a benchmark suite, the \ac{E3S}~\cite{e3s}.
Based on data from the Embedded Microprocessor Benchmark Consortium, the suite provides task graphs and processor execution times for applications from multiple embedded domains.
In total, the \ac{E3S} boasts $20$ benchmarks from $5$ different domains, with up to $9$ tasks per benchmark.
Table~\ref{tab:e3s} summarizes the applications.

\begin{table*}[t]
  %\footnotesize
  \caption{Summary of applications in the \ac{E3S}}
  \begin{center}
    \begin{tabular}{lll}
      %\hline
      Domain & No. of task graphs & tasks per graph\\
      \hline
      auto-indust. & 4 & 4-9\\
      networking & 4 & 1-4\\
      telecom & 9 & 2-6\\
      consumer & 2 & 5-7\\
      office-automation & 1 & 5\\
    \end{tabular}
    \label{tab:e3s}
  \end{center}
  \vspace{-0.5cm}
\end{table*}

The benchmark suite is pretty dated, being over $20$ years old at the time of this writing.
Unfortunately, benchmarks are generally scarce.
The methods investigated in this thesis here have more to do with the trends than the actual numbers, which is why using such a dated benchmark suite is still adequate.
We expect the relative performance of mapping algorithms on the \ac{E3S} benchmarks to be similar to that on present and future applications.

A significant focus of the methods we will evaluate with these benchmarks is on the multicore architectures.
For this, we use the same method as in~\cite{weichslgartner2014daarm,schwarzer2017symmetry}.
We use the architecture topology of a modern multicore, including the frequencies, as well as the memory subsystem with its latency and bandwidth, and scale the numbers from the \ac{E3S} for each of the cores of the modern multicore.
This gives a realistic scenario, albeit not simulating a concrete instance of the architecture.
In~\cite{schwarzer2017symmetry} the authors do this to create architectures with a regular mesh structure, with less realistic topologies like heterogeneous meshes with randomly placed cores.
Instead, we use the topologies from actual existing systems like the HAEC~\cite{HAEC} or the Kalray MPPA3 Coolidge~\cite{coolidge} and map the processors in these architectures to those in the benchmark suite.