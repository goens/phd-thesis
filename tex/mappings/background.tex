\Blindtext[3]

\section{A brief survey of models of Computation}
\label{sec:mocs}

\begin{figure}[h]
	\centering
   \resizebox{0.55\textwidth}{!}{\begin{tikzpicture}%[fill opacity=0.3]
%circles
\node[ellipse, fill=green!20, minimum width = 9cm,minimum height = 5cm,draw] (ddf) at (0,3) {};
\node[ellipse, fill=green!60, minimum width = 6.2cm,minimum height = 4.5cm,draw] (kpn) at (1,3) {};
\node[ellipse, fill=green!50, minimum width = 5.5cm,minimum height = 4.0cm,draw] (kmq) at (1,2.8) {};
\node[ellipse, fill=green!70, minimum width = 4.8cm,minimum height = 3.5cm,draw] (csdf) at (1,2.6) {};
\node[ellipse, fill=green!30, minimum width = 3cm,minimum height = 3cm,draw] (sdf) at (1.5,2.5) {SDF};
\node[ellipse, fill=green!40, minimum width = 1.0cm,minimum height = 1.0cm,draw] (hsdf) at (1.5,3.4) {HSDF};

%labels
\node (lbl-ddf) at (-3,3) {DDF};
%\node (lbl-sdf) at (0,1.5) {SDF};
%\node (lbl-hsdf) at (0,0) {HSDF};
\node (lbl-kmq) at (1,4.5) {KMQ};
\node (lbl-sadf) at (-0.6,2.5) {CSDF};
\node (lbl-kpn) at (1,5.0) {KPN};
  
\end{tikzpicture}
}
	\caption{Relationships between different dataflow models of computation.}
	\label{fig:dataflow_mocs}
\end{figure}

\subsection{Kahn Process Networks}

In 1970, Dana Scott proposed a mathematical theory of computation~\cite{scott1970} based on what are now called (Scott) domains\footnote{sometimes also called algebraic semilattice} and the Scott-topology. 
Two ideas are central in Scott's formalization. The first is a method for capturing \emph{partial} computations, i.e. computations that have advanced but not finished yet.
The second idea is that of modeling a computation as a continuous function between such domains, where a properly notion of continuity (in the Scott topology) models causality in the computation.
Scott's semantics allowed to capture the process of a computation, but not the internals, which are abstracted away by the function. 

A Domain is a particular type of \ac{poset} ... 

For example ...


Scott's computation model implicitly assumed a sequential computation and Scott-continuous functions are a powerful method for describing partial sequential computations.
Can we also use this model to describe parallel computation?
Gilles Kahn did precisely this, four years after Scott published his mathematical theory of computation. 
He used the formalism of Scott to define a model of parallel computation, based on what he coined as process networks, now known as Kahn Process Networks.

The basic idea to generalize the Scott theory of computation is simple
