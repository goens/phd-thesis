At the basis of software synthesis are concurrent models of computation. The process derives an efficient execution to a concrete architecture from a computation expressed in an abstract model.
In order to understand software synthesis thus, we first need to understand the underlying models of computation.
Closely related to these concurrent models of computation are execution traces, which capture a concrete execution for some input.
However, going from such abstract models to a concrete execution also requires an understanding of the target hardware architecture.
The relationship between the two can be captured in a mapping.
This chapter considers all of these aspects with their corresponding models, and the relationship between them.
It is the central piece of background theory required for the methods presented in this thesis.

\subsection{Models of Computation}
\label{sec:mocs}

In his seminal paper~\cite{turing1936computable} in 1936, Alan Turing proposed a ``computing machine''\footnote{now known as Turing machine}.   
While his machine was motivated by a person doing computations, he intended to capture the very notion of compatibility by it: namely what it is possible to compute at all.
He was modeling computation.
Two additional such models of computation existed at the time, the $\lambda$-calculus as proposed by Alonzo Church that same year~\cite{church1936unsolvable}, and the concept of general recursive functions due to Herbrand and Gödel, developed by Kleene~\cite{kleene1936recursive}.
These three equivalent models~\cite{turing1937comuptability} were the original models of computation.
They are equivalent in the sense that they define the same notion of what is computable.
To an extent these models were not concerned with \emph{how} to (efficiently) compute something, but rather, \emph{what} we can compute and what not.
Since then, with the revolution of digital computers, the interest increasingly shifted to care about \emph{how} we can compute.
This spawned a much larger amount of models of computation at different levels of abstraction.

In 1972 Karp\cite{karp1972reducibility} kickstarted the field of computational complexity by identifying many problems that were equivalently difficult to compute, the class of NP-complete problems.
Computational complexity relies on the fact that the asymptotic behavior of the number of steps of an algorithm, as a function of the input (size), is invariant when changing between these models of computation.
Around the same time, in 1970, Dana Scott proposed a mathematical theory of computation~\cite{scott1970} based on what are now called (Scott) domains\footnote{sometimes also called algebraic semilattice} and the Scott-topology. 
Two ideas are central in Scott's formalization. The first is a method for capturing \emph{partial} computations, i.e. computations that have advanced but not finished yet.
The second idea is that of modeling a computation as a continuous function between such domains, where a properly notion of continuity (in the Scott topology) models causality in the computation.
Scott's semantics allowed to capture the process of a computation, but not the internals, which are abstracted away by the function. 

The question of \emph{how} we compute can be modeled in different ways by complexity asymptotics or partial computations in the Scott formalism, but some aspects are still left unmodeled.
A significant such aspect not taken into account by these models is \emph{where} we are computing.
The theory of distributed computation was growing, with models like Petri Nets~\cite{petri1962nets} or seminal work like Lamport's on clocks and ordering of events~\cite{Lamport1978time}.
These models deal with properties of a computing system that has physically separate parts which split and distribute the computational load.
However, the focus of the models is the system doing the computation, not the computation itself.

In this thesis we are mostly interested in concurrent models of computation. 
Such models abstract away the (distributed) computing system and focus on the computation itself. 
They consider and express concurrency in the computation, which can be exploited for parallel or asynchronous execution.
The rest of this section will survey some of the most important concurrent models of computation, before moving on to architectures, traces and mappings.


\begin{figure}[h]
	\centering
   \resizebox{0.55\textwidth}{!}{\begin{tikzpicture}%[fill opacity=0.3]
%circles
\node[ellipse, fill=green!20, minimum width = 9cm,minimum height = 5cm,draw] (ddf) at (0,3) {};
\node[ellipse, fill=green!60, minimum width = 6.2cm,minimum height = 4.5cm,draw] (kpn) at (1,3) {};
\node[ellipse, fill=green!50, minimum width = 5.5cm,minimum height = 4.0cm,draw] (kmq) at (1,2.8) {};
\node[ellipse, fill=green!70, minimum width = 4.8cm,minimum height = 3.5cm,draw] (csdf) at (1,2.6) {};
\node[ellipse, fill=green!30, minimum width = 3cm,minimum height = 3cm,draw] (sdf) at (1.5,2.5) {SDF};
\node[ellipse, fill=green!40, minimum width = 1.0cm,minimum height = 1.0cm,draw] (hsdf) at (1.5,3.4) {HSDF};

%labels
\node (lbl-ddf) at (-3,3) {DDF};
%\node (lbl-sdf) at (0,1.5) {SDF};
%\node (lbl-hsdf) at (0,0) {HSDF};
\node (lbl-kmq) at (1,4.5) {KMQ};
\node (lbl-sadf) at (-0.6,2.5) {CSDF};
\node (lbl-kpn) at (1,5.0) {KPN};
  
\end{tikzpicture}
}
	\caption{Relationships between different dataflow models of computation.}
	\label{fig:dataflow_mocs}
\end{figure}

\subsection{Partial Computation: Scott Domains}

When Dana Scott proposed his mathematical theory of computation~\cite{scott1970}, he used the term mathematical to contrast it with operational computation.
In practice, the steps of a computation are defined by the \ac{ISA} of the machine executing them.
Most people don't write programs directly for the \ac{ISA}, however. They write them in an abstract programming language, which is translated by a compiler into machine instructions.
Thus, in practice, the implementation of a compiler is what defines the (operational) semantics of programs.
Scott's theory had the ambitious goal of being an abstraction that sat between these operational semantics and the abstract notions of computability of e.g. Church or Turing.
He intended to abstract away the arbitrary implementation choices that were necessary but did not change the essence of the execution.
While today his model is not the single established abstract model of semantics he sought out to define, it introduced several important ideas and mathematical structures to models of computation.
In particular, a crucial abstraction introduced by his theory is that of partial computation.
His theory makes it possible to express a computation as a series of partial results, without regarding the actual implementation of these.
We will now introduce the basics of Scott's mathematical theory of computation.


A Domain is a particular type of \ac{poset} ... 

For example ...


Scott's computation model implicitly assumed a sequential computation and Scott-continuous functions are a powerful method for describing partial sequential computations.
Can we also use this model to describe parallel computation?
Gilles Kahn did precisely this, four years after Scott published his mathematical theory of computation. 
He used the formalism of Scott to define a model of parallel computation, based on what he coined as process networks, now known as Kahn Process Networks.

The basic idea to generalize the Scott theory of computation is simple



\subsection{Concurrent Computation: Kahn Process Networks}
Discuss related: Hewitt-Agha actor model\index{actor model}~\cite{DBLP:conf/ijcai/HewittBS73,Agha:86:Actors}, Petri Nets\index{Petri nets}~\cite{petri1962nets}, Process Calculi\index{Process Calculi} (\index{Process Calculi ! Pi-Calculus}$\Pi$-calculus, \index{Process Calculi ! CSP}\ac{CSP})
\Blindtext[10]\index{Kahn Process Networks}\ac{KPN}~\cite{kahn74}

\subsection{Dataflow Models of Computation}
Dennis dataflow \index{Dataflow ! Dennis}~\cite{dennis1974first,dennis1986data}
\cite{Parks:M95/105}
Dataflow process networks~\cite{lee1995dataflow,lee_matsikoudis_semantics}
\ac{CSDF}\index{Dataflow ! CSDF}~\cite{bilsen1996cycle}
\ac{SADF}\index{Dataflow ! SADF}~\cite{theelen2006scenario}
\ac{SDF}\index{Dataflow ! SDF}~\cite{lee1987sdf}
\Blindtext[10]

\subsection{Execution Traces}
Models of Computation with mathematical semantics, in the sense of Scott, abstract away the concrete implementation of individual steps in a computation.
This is also the case for Kahn Process Networks or even Dataflow models, when considered as mathematical semantics of computation. 
Even so, the execution of a computation can be thought of as a series of steps or partial computations that eventually yield the final result.
These series, which is commonly referred to as execution trace, can be captured as a sequence of steps, e.g. as the element of a Scott Domain. 
Abstract computations, modeled as Scott-continuous functions, can can make computations of arbitrary length.
This is modeled by (countably) infinite sequences in $D^\omega \setminus D^*$.
A concrete execution, on the other hand, always has a finite length.
It always resides in $D^*$.
For a (Scott-continuous) function, this sequence can be modeled as a finite string in the computation domain.

In a concurrent execution, multiple entities concurrently execute steps.
As modeled by Kahn, these entities all implement individual (Scott-continuous) functions.
As such, there is not a unique series of steps that can be said to be the execution trace of the computation.
To see this, consider the following example:
\begin{ex}
  Todo
\end{ex}

TODO: discuss example

In the distributed case thus, the execution traces are in fact equivalence classes of strings.
We define this more formally, following~\cite{mazurkiewicz1995introduction}, the first chapter of~\cite{diekert1995book}.
Let $D$ be a finite set, like an alphabet $\Sigma$ or data type $D = \Cup_{i \in \mathcal{I}} D_i$ be a data type for a \ac{KPN}.
Let $\Delta$ be a symmetric, reflexive relation on $D$, which we call a dependency\index{dependency}.
This means that if $(a,b) \in \Delta$, we have $(b,a) \in \Delta$ and also $(a,a) \in Delta$ for all $a \in D$. 
With $\Delta$ we define an additional relation in $D$, namely $I := (D \times D) \setminus \Delta$.
We call $I$ the induced independency\index{independency}. 
We define an equivalence relation $\sim_I$ on the monoid $D^*$ (with respect to concatenation) as follows:
We require for $a,b \in I$ then $ab \sim_I ba$. The relation $\sim_I$ is defined as the least congruence that satisfies this requirement.
Note that a congruence is an equivalence relation that respects the algebraic structure, in this case the monoid structure of the concatenation operation.
We call the equivalence classes of $D^*/{\sim_I}$ traces. 
By definition, the concatenation operation on $D^*$ factors over the equivalence relation $\sim_I$,
and thus $D^*/{\sim_I}$ defines a monoid (with identity element $[\epsilon]_{\sim_i}$, where $\epsilon \in D^*$ is the empty string).
We call this the Trace Monoid\index{trace monoid}, $T(D)$.
We care about the algebraic structure of a monoid since it is central to the definition of Scott-continuity.

\begin{ex}
  Todo (based on example above)
\end{ex}

There are two additional equivalent definitions of this monoid as histories and dependence graphs.
We present histories here, as they are easier for the intuition.
Instead of a single alphabet (or set of types) $D$, we have a finite set of alphabets $D := (D_i), i \in \mathcal{I}$, where $\mathcal{I}$ is a finite index set.
We can think of the indices as corresponding to the processes or actors in the system, and the alphabets $D_i$ to the alphabets of these individual entities.
If we think of the individual entities as computing some (Scott-continuous) function, their execution trace will be a unique string $a_i \in D_i^*$ (recall that concrete executions are finite).
Since, in general, these entities do not compute independently, they have common synchronization points.
These synchronization points are abstractly modeled in the computation alphabet by mutual elements in $D_i \cap D_j$ for two entities $i,j \in \mathcal{I}$.
We can define a monoid, the product monoid~\index{product monoid} $P(D)$, by component-wise concatenation of the strings: $(a_i)_i (b_i)_i = (a_ib_i)_i$ for all $i \in \mathcal{I}$.
However, not every such a string product can be the history of a system.
The synchronization points of different subsystems should be consistent with each other. Consider the following example:
\begin{ex}
  \label{ex:inconsistency_history}
TODO: slight variation that shows an element of $P(D) \setminus H(D)$
\end{ex}
To avoid this, we want to ensure histories are consistent. For this, we define elementary histories~\index{elementary history} as follows:
For any $a in \Cup_{i \in \mathbb{I}} D_i$, the elementary history of $a$ is the tuple $(a_i)_{i \in \mathbb{I}}$, with $a_i = \left{ \begin{array}{ll} a, & \text{ if } a \in D_i, \\ \epsilon, & \text{ otherwise. }\end{array}\right.$
Again, $\epsilon$ represents the empty string.
The monoid generated by all elementary histories for elements in $\Cup_{i \in \mathbb{I}} D_i$ is called the history monoid $H(D)$, and is a submonoid of $P(D)$.
If we examine the definition, it is not difficult to convince ourselves that these are precisely the histories which avoid inconsistencies like those of Example~\ref{ex:inconsistency_history}.

\begin{ex}
  Todo (based on example above)
\end{ex}

We can go from a trace to a history by the morphism $\pi: T(\Cup_{i \in \mathcal{I}}) \rightarrow H(D), a \mapsto (\pi_i(a))_i, i \in \mathal{I}$, where $\pi_i$ is the projection $\Cup_{i \in \mathcal{I}}D_i \rightarrow D_i$.
Here, for the trace monoid $T(D)$ we define the dependencies to be $\Cup_{i \in \mathcal{I}}\D_i \times D_i$. 
This is not just a morphism, but in fact an isomorphism: See Theorem 1.5.4 of~\cite{mazurkiewicz1995introduction}.
Thus, the two concepts are equivalent.
For the rest of this thesis we will use the terms traces and histories interchangeably.


Traces, and equivalently histories, can be used to describe the concrete computations in concurrent systems like those described by a \ac{KPN}.
They are also well-suited (and well-suited) to model these systems in the context of process calculi, like \ac{CSP}.
However, an important observation is the converse: a concrete execution of a \ac{KPN} is determined uniquely by its history.
Moreover, any concrete implementation of the \ac{KPN} realizing the same execution will have the same history: the history is an invariant of the abstract execution model.
It captures the concurrent essence of the concrete computation.

\subsection{Architecture Models}
\cite{pelcat2015models}.
\Blindtext[10]

\subsection{The Mapping Problem}
\cite{singh2013mapping}
A lot more...
\Blindtext[10]
\subsection{Limits of the Model}
\Blindtext[10]

