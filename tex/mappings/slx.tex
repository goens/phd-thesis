Many flows exist that enable model-based design in a software synthesis flow.
SystemCoDesigner~\cite{haubelt2008systemcodesigner} is based on SystemC and aimed at \acsp{FPGA}, as is the case with CAPH~\cite{serot2013caph}, which is based on dataflow and the actor model.
Although the flows are based on \acp{MoC}, their goal is not software but rather an \acsp{FPGA} implementation, and as such these flows are closer to \ac{HLS} than the rest. 
Coincidentally, the term software synthesis is inspired on the much better-known \ac{HLS}.

Also based on a more general dataflow model is the Turnus~\cite{casale2013turnus} flow.
It builds on top of RVC-CAL, which is in turn based on the CAL actor language~\cite{cal}.

More specific is the \ac{\SDFFF}\cite{sdf3} framework, which does much more than generating random \ac{SDF} graphs.
As a software synthesis tool~\cite{stuijk2010predictable}, it focuses on the more restricted \ac{SDF} \ac{MoC}, allowing much more sophisticated analysis of the applications.
Similarly, PREESM~\cite{pelcat2014preesm} works with parametrized extensions of \ac{SDF}~\cite{desnos2013pimm} that provide a greater trade-off between expressiveness and analyzability.

On the other side of the \ac{MoC} spectrum, many related flows use \ac{KPN}.
The static-mapping-based flows of \ac{DOL}\cite{thiele2007dol}, Sesame~\cite{pimentel2006systematic} or \ac{MAPS}~\cite{maps} use different levels of abstraction to derive an efficient execution form a \ac{KPN}-based application description.

Going beyond static mapping, the DAARM~\cite{weichslgartner2014daarm} flow maps dataflow applications using a hybrid approach.
Similarly, the work of~\cite{quan2015hybrid} extends the Sesame approach to hybrid mappings, and Spider~\cite{heulot2014spider} extends the work of PREESM to hybrid mappings.

This thesis and the contributions included in it are \textbf{not} aimed at proposing (yet another) software synthesis design flow.
Instead, we propose methods to improve existing flows, with the ambitious goal of being general enough that the improvements would benefit most of the flows discussed. 
Perhaps a good way to think of this is that just as these flows help users write more efficent applications, we aim to help the flow designers improve their flows.
For chapters~\ref{chap:mapping_structures} and \ref{chap:mapping_applications}, we focus on one flow to do this.
We choose the \ac{MAPS} flow~\cite{maps}, which resulted in the commercial spinoff Silexica, who kindly allowed us to use their tools for our research.
Some contributions, on the other hand, go beyond these flows. This is particularly the case in chapters~\ref{chap:mocs} and~\ref{chap:pl}, and in part Chapter~\ref{chap:benchmarking} 

\subsection{The \ac{MAPS} Maping flow}
