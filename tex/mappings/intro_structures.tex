The space of mappings in the software synthesis flow we described has a rich mathematical structure.
This chapter aims to explore and expose that structure, at least in part.
We will consider two main aspects of the mathematical structure hidden within the simple notion of a mapping, namely the inherent symmetry, and the degrees of similarity between mappings.
We will consider how to extract this structure in a computationally-efficient fashion, and how it can be exposed to tools that aim to exploit it, in different representations.

This thesis focuses primarily on a view of the mapping problem centered on computation, instead of data.
In many cases, with the increasing discrepancy between execution frequency and memory access times (cf. Figure~\ref{fig:multicore_era}) this view is not ideal.
The problem space of data allocation is usually more clearly structured and can be modeled better. 
For example, we worked on \ac{ILP}-based methods to describe and optimize memory allocation~\cite{odendahl_date14,odendahl15,goens_jsa16}.\index{\acl{ilp}}
We omit this work from this thesis for space reasons, to keep the scope manageable.

We also omit work on emerging memory technologies, concretely \ac{RTM}, where we used similar \ac{ILP}-based models and other meta-heuristics like genetic algorithms or domain-specific heuristics to optimize data placement~\cite{khan_date20}.\index{\acl{RTM}}\index{emerging memory technologies}
