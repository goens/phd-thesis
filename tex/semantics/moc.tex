In his seminal paper~\cite{turing1936computable} in 1936, Alan Turing proposed a ``computing machine''\footnote{now known as Turing machine}.   
While his machine was motivated by a person doing computations, he intended to capture the very notion of compatibility by it: namely what it is possible to compute at all.
He was modeling computation.
Two additional such models of computation existed at the time, the $\lambda$-calculus as proposed by Alonzo Church that same year~\cite{church1936unsolvable}, and the concept of general recursive functions due to Herbrand and Gödel, developed by Kleene~\cite{kleene1936recursive}.
These three equivalent models~\cite{turing1937computability} were the original models of computation.
They are equivalent in the sense that they define the same notion of what is computable.
To an extent these models were not concerned with \emph{how} to (efficiently) compute something, but rather, \emph{what} we can compute and what not.
Since then, with the revolution of digital computers, the interest increasingly shifted to care about \emph{how} we can compute.
This spawned a much larger amount of models of computation at different levels of abstraction.

In 1972 Karp\cite{karp1972reducibility} kickstarted the field of computational complexity by identifying many problems that were equivalently difficult to compute, the class of NP-complete problems.
Computational complexity relies on the fact that the asymptotic behavior of the number of steps of an algorithm, as a function of the input (size), is invariant when changing between these models of computation.
Around the same time, in 1970, Dana Scott proposed a mathematical theory of computation~\cite{scott1970} based on what are now called (Scott) domains\footnote{sometimes also called algebraic semilattice} and the Scott-topology. 
Two ideas are central in Scott's formalization. The first is a method for capturing \emph{partial} computations, i.e. computations that have advanced but not finished yet.
The second idea is that of modeling a computation as a continuous function between such domains, where a properly notion of continuity (in the Scott topology) models causality in the computation.
Scott's semantics allowed to capture the process of a computation, but not the internals, which are abstracted away by the function. 

The question of \emph{how} we compute can be modeled in different ways by complexity asymptotics or partial computations in the Scott formalism, but some aspects are still left unmodeled.
A significant such aspect not taken into account by these models is \emph{where} we are computing.
The theory of distributed computation was growing, with models like Petri Nets~\cite{petri1962nets} or seminal work like Lamport's on clocks and ordering of events~\cite{Lamport78time}.
These models deal with properties of a computing system that has physically separate parts which split and distribute the computational load.
However, the focus of the models is the system doing the computation, not the computation itself.

In this thesis we are mostly interested in concurrent models of computation. 
Such models abstract away the (distributed) computing system and focus on the computation itself. 
They consider and express concurrency in the computation, which can be exploited for parallel or asynchronous execution.
The rest of this section will survey some of the most important concurrent models of computation, before moving on to architectures, traces and mappings.


\begin{figure}[h]
	\centering
   \resizebox{0.55\textwidth}{!}{\begin{tikzpicture}%[fill opacity=0.3]
%circles
\node[ellipse, fill=green!20, minimum width = 9cm,minimum height = 5cm,draw] (ddf) at (0,3) {};
\node[ellipse, fill=green!60, minimum width = 6.2cm,minimum height = 4.5cm,draw] (kpn) at (1,3) {};
\node[ellipse, fill=green!50, minimum width = 5.5cm,minimum height = 4.0cm,draw] (kmq) at (1,2.8) {};
\node[ellipse, fill=green!70, minimum width = 4.8cm,minimum height = 3.5cm,draw] (csdf) at (1,2.6) {};
\node[ellipse, fill=green!30, minimum width = 3cm,minimum height = 3cm,draw] (sdf) at (1.5,2.5) {SDF};
\node[ellipse, fill=green!40, minimum width = 1.0cm,minimum height = 1.0cm,draw] (hsdf) at (1.5,3.4) {HSDF};

%labels
\node (lbl-ddf) at (-3,3) {DDF};
%\node (lbl-sdf) at (0,1.5) {SDF};
%\node (lbl-hsdf) at (0,0) {HSDF};
\node (lbl-kmq) at (1,4.5) {KMQ};
\node (lbl-sadf) at (-0.6,2.5) {CSDF};
\node (lbl-kpn) at (1,5.0) {KPN};
  
\end{tikzpicture}
}
	\caption{Relationships between different dataflow models of computation.}
	\label{fig:dataflow_mocs}
\end{figure}

\subsection{Partial Computation: Scott Domains}

When Dana Scott proposed his mathematical theory of computation~\cite{scott1970}, he used the term mathematical to contrast it with operational computation.
In practice, the steps of a computation are defined by the \ac{ISA} of the machine executing them.
Most people don't write programs directly for the \ac{ISA}, however. They write them in an abstract programming language, which is translated by a compiler into machine instructions.
Thus, in practice, the implementation of a compiler is what defines the (operational) semantics of programs.
Scott's theory had the ambitious goal of being an abstraction that sat between these operational semantics and the abstract notions of computability of e.g. Church or Turing.
He intended to abstract away the arbitrary implementation choices that were necessary but did not change the essence of the execution.
While today his model is not the single established abstract model of semantics he sought out to define, it introduced several important ideas and mathematical structures to models of computation.
In particular, a crucial abstraction introduced by his theory is that of partial computation.
His theory makes it possible to express a computation as a series of partial results, without regarding the actual implementation of these.
We will now introduce the basics of Scott's mathematical theory of computation.


A Domain is a particular type of \ac{poset} ... 

For example ...


Scott's computation model implicitly assumed a sequential computation and Scott-continuous functions are a powerful method for describing partial sequential computations.
Can we also use this model to describe parallel computation?
Gilles Kahn did precisely this, four years after Scott published his mathematical theory of computation. 
He used the formalism of Scott to define a model of parallel computation, based on what he coined as process networks, now known as Kahn Process Networks.

The basic idea to generalize the Scott theory of computation is simple



\subsection{Concurrent Computation: Kahn Process Networks}
Discuss related: Hewitt-Agha actor model\index{actor model}~\cite{DBLP:conf/ijcai/HewittBS73,Agha:86:Actors}, Petri Nets\index{Petri nets}~\cite{petri1962nets}, Process Calculi\index{Process Calculi} (\index{Process Calculi ! Pi-Calculus}$\Pi$-calculus, \index{Process Calculi ! CSP}\ac{CSP})
\Blindtext[10]\index{Kahn Process Networks}\ac{KPN}~\cite{kahn74}

\subsection{Dataflow Models of Computation}
Dennis dataflow \index{Dataflow ! Dennis}~\cite{dennis1974first,dennis1986data}
\cite{Parks:M95/105}
Dataflow process networks~\cite{lee1995dataflow,lee_matsikoudis_semantics}
\ac{CSDF}\index{Dataflow ! CSDF}~\cite{bilsen1996cycle}
\ac{SADF}\index{Dataflow ! SADF}~\cite{theelen2006scenario}
\ac{SDF}\index{Dataflow ! SDF}~\cite{lee1987sdf}
\Blindtext[10]



