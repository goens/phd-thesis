In this Section we introduce the Ohua programing paradigm~\cite{ertel_phdthesis}.
The Ohua programming model by S. Ertel and others is a powerful model of implicit parallelism which can be used to express parallelism at a language level without explict costructions like threads and locks. 
This model is not part of the original contribution of this thesis. However, it is central to several distinct original contributions of the thesis, and we will introduce it as background material.

Ohua is best understood by diving directly into examples. Consider the code in Figure~\ref{fig:ohua_example}.
\begin{figure}[h]
	\centering
   \resizebox{0.55\textwidth}{!}{\begin{tikzpicture}
\draw (0,0) rectangle (5,5);
\draw (2.5,2.5) node {Placeholder};
\end{tikzpicture}
}
	\caption{A placeholder picture.}
	\label{fig:ohua_example}
\end{figure}

\todo{discuss example}.

The example in Figure~\ref{fig:ohua_example} can be transformed into a dataflow graph for execution. Figure~\ref{fig:ohua_example_df} depicts this example as a dataflow graph.
\todo{discuss dataflow graph}.

This duality between code and dataflow graphs is the core concept behind Ohua. More generally, Ohua defines a language...
\todo{discuss ohua formally}

\subsection{Stateful Functions}

\subsection{Dataflow Execution}

\subsection{$\lambda$-calculus-based Language}
