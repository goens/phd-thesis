In his seminal paper in 1936, Alan Turing proposed what he called a ``computing machine''\footnote{Now known as a Turing machine}.   
While his machine was motivated by a person doing computations, he intended to capture the very notion of compatibility by it: namely what is possible to compute at all.
He was modeling computation.
Two additional such models of computation existed at the time, the $\lambda$-calculus as proposed by Alonzo Church that same year~\cite{church1936unsolvable}, and the concept of general recursive functions due to Herbrand and Gödel, developed by Kleene~\cite{kleene1936recursive}.
These three equivalent models~\cite{turing1937computability} were the original models of computation.
They are equivalent in the sense that they define the same notion of what is computable.
To an extent, these models were not concerned with \emph{how} to (efficiently) compute something, but rather, \emph{what} we can compute and what not.
Since then, with the revolution of digital computers, the interest increasingly shifted to care about \emph{how} we can compute.
This spawned a much larger amount of models of computation at different levels of abstraction.

In 1972 Karp\cite{karp1972reducibility} kickstarted the field of computational complexity by identifying many problems that were equivalently difficult to compute, the class of NP-complete problems.
Computational complexity relies on the fact that the asymptotic behavior of the number of steps of an algorithm, as a function of the input (size), is invariant when changing between these models of computation.
Around the same time, in 1970, Dana Scott proposed a mathematical theory of computation~\cite{scott1970} based on what are now called (Scott) domains\footnote{Also called $\omega$-complete partial orders~\cite{gunter}, and closely related to algebraic lattices.} and the Scott-topology. 
Two ideas are central in Scott's formalization. The first is a method for capturing \emph{partial} computations, i.e. computations that have advanced but not finished yet.
The second idea is that of modeling a computation as a continuous function between such domains, where a properly notion of continuity (in the Scott topology) models causality in the computation.
Scott's semantics allowed to capture the process of a computation, but not the internals, which are abstracted away by the function. 

The question of \emph{how} we compute can be modeled in different ways by complexity asymptotics or partial computations in the Scott formalism, but some aspects are still left unmodeled.
A significant such aspect not taken into account by these models is \emph{where} we are computing.
The theory of distributed computation was growing, with models like Petri Nets~\cite{petri1962nets} or seminal work like Lamport's on clocks and ordering of events~\cite{Lamport78time}.
These models deal with properties of a computing system that has physically separate parts which split and distribute the computational load.
However, the focus of the models is the system doing the computation, not the computation itself.

In this thesis we are mostly interested in concurrent models of computation. 
Such models abstract away the (distributed) computing system and focus on the computation itself. 
They consider and express concurrency in the computation, which can be exploited for parallel or asynchronous execution.
