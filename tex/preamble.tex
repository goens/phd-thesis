\usepackage{makeidx}
\usepackage{hyperref}
\usepackage{todonotes}
\usepackage{amsmath}
\usepackage{amssymb}
\usepackage{graphicx}
\usepackage{tikz}
\usepackage{pgfgantt}
\usepackage{acronym}
%\usepackage[english]{babel}
\usepackage{smartdiagram}
\usepackage{epigraph}
\usepackage{minted}
\usepackage{xspace}
\usepackage{ifthen}
\usepackage[nounderscore]{syntax}
%\renewcommand{\blindtext}[1][]{blindtext:#1}
%\renewcommand{\Blindtext}[1][]{Blindtext:#1}

\usepackage{amsthm}
\newenvironment{bew}{\begin{proof}[Proof]}{\end{proof}}
\theoremstyle{definition}
\newtheorem{Satz}{Satz}[section]
\newtheorem{thm}[Satz]{Theorem}
\newtheorem{ex}[Satz]{Example}
\newtheorem{cor}[Satz]{Corollary}
\newtheorem{algorithm}[Satz]{Algorithm}
\newtheorem{prop}[Satz]{Proposition}
\newtheorem{rem}[Satz]{Remark}
\newtheorem{defn}[Satz]{Definition}
\newtheorem{lem}[Satz]{Lemma}

\makeindex 

\usepackage[
backend=biber,
style=alphabetic,
maxbibnames=40,
sorting=ynt
]{biblatex}
\addbibresource{aux/references.bib}
\addbibresource{aux/mios.bib}

\newcommand \drawpe[2] {
  \draw ({#1+0}, {#2+0}) rectangle ++(2,2);
  \draw ({#1+2},{#2+2}) -- ++(0.5,0.5);
  \draw ({#1+2.5},{#2+2.5}) rectangle ++(1,1) ++(-0.5,-0.5) node[font=\Large] {R};
}

\newcommand \drawmesh[4] {
  \foreach \x in {0,...,{#1 - 1}}{
    \foreach \y in {0,...,{#2 - 1}}{
      \drawpe{{#3+\x*4}}{{#4+\y*4}}
    }
  }
}
  

\def\dom{\operatorname{dom}}
\def\cod{\operatorname{cod}}
\def\PE{\operatorname{PE}}
\def\AutSemi{\operatorname{AutSemi}}
\def\Aut{\operatorname{Aut}}
\newcommand{\Goens}[1]{\textbf{#1}} %hack to make bibliography work with alpha

%https://tex.stackexchange.com/questions/354715/r-tikzdevice-relative-directories
\newcommand\inputTikz[2][generated]{ 
    \let\pgfimageOld\pgfimage% 
    \renewcommand{\pgfimage}[2][]{\pgfimageOld[##1]{#1/##2}}% 
    \input{#1/#2}% 
    \let\pgfimage\pgfimageOld% 
}