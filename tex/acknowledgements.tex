First and foremost, I thank my advisor Jeronimo Castrillon.
I consider him to have been both a mentor and a friend during the time I've spent working on this thesis.
His advice shaped my research and this thesis would not exist without his guidance and help.

I also want to thank my current and former colleagues and co-authors at the chair for compiler construction:
Justus Adam, Hasna Bouraoui, Alexander Brauckmann, Sebastian Ertel, Fazal Hameed, Gerald Hempel, Sven Karol, Asif Khan, Robert Khasanov, Nesrine Khouzami, Christian Menard, Norman Rink, Julian Robledo, Lars Schütze and Felix Wittwer.
Thank you for creating a great environment to learn and work together, for countless discussions and insights, for your patience with my insistance on going to Zeltmensa and the great discussions that arose there, and for offering my comradery and friendship. 

I want to thank the students working with me, who've helped me with work, some of whom have become colleagues in the meantime, and from all of whom I've also learned a great deal.
Concretely, thank you, Alexander Brauckmann, Christian Menard, Marcus Rossel, Alexander Thierfelder, Felix Teweleitt and Markus Walter.

During my Ph.D I had the opportunity to visit Andy Pimentel at the University of Amsterdam, where I was warmly received by him, Simon Polstra and the rest of the group.
Thank you for welcoming me and for a fruitful collaboration. I'd also like to thank the HiPEAC project for funding this visit through a collaboration grant.

I also had the opportunity to visit Edward Lee at the University of Califonria at Berkeley. There, Matt Webber and Gil Lederman received me in their office, where I felt very welcome, like any other colleague.
I want to thank both, as well as Marten Lohstroh, all of whom I had great discussions with, and who made my visit at Berkeley extremely fruitful.
A special thank you also goes to Mary Stewart for helping sort out everything there, even to the point of making sure I had something to eat at the group lunches. 
Most of all, I would like to thank Edward Lee for accepting me to visit his group and taking the time to talk with me regularly.
This visit was a pivotal point in my Ph.D. and I really appreciated everything and everyone.
Outside the academic realm, I want to thank Giulia Leggett for making this visit extremely enriching also from a personal point of view.
I also want to thank the German foreign exchange service DAAD and specifically the FIT Weltweit project, as well, the \ac{cfaed} cluster of excellence, for helping me finance this visit.

I also want to thank the rest of my co-authors. Chris Cummins and Hugh Leather for being so open in our collaboration and for their hospitality in Edingburgh. 
To Max Odendahl, for trusting in my abilities while knowing me only on a personal level, and introducing me to the field. Without him, I would never have come to this area of research.
To everyone in the cfead Orchestration path, for sharing a vision with me and constructive retreats.
I also thank Marcus Hähnel and Till Smejkal for a very successful collaboration what started the \acs*{TETRiS} project.
Thanks to Josefine Asmus and Ivo Sbalzarini for collaboration on the work on design centering, which was very insightful.
I also thank Sergio Siccha, for taking our friendly discussions so seriously that we ended up collaborating in the mapping symmetries work.
Thanks also to Robert Wittig for a beachside discussion at Samos that led to a collaboration on the model-based approaches to 5G.

I started this Ph.D. at the \ac{cfaed} cluster of excellence, which provided funding and a great academic environment.
I want to thank everyone at the program office for helping me throughout this time, as well as my thesis advisory committee, Jeronimo Castrillon, Christel Baier and Hermann Härtig.
I also want to thank Conny Okuma for her patience throughout the years with my incomplete formularies and late handing over of documents.
Thanks as well to the German Research Foundation DFG for funding me after \ac{cfaed}.

The final phase of my Ph.D. was mainly funded by the Studienstiftung des deutschen Volkes.
Besides financial support also provided me with an excellent offer of intellectual complementary opportunities.
Thank you for this opportunity, and thanks to Maike Lieser for helping me apply to this scholarship, I am sure I would not have received it without her help.
I also want to thank her for everything else, as she was probably the biggest positive influence in my life during the time of my Ph.D.

I want to thank everyone at TEDxDresden and everyone from animal rights activism for giving me meaningful projects to do with my life besides my research. 
Also to everyone at Bodyworks and Basketball Club Dresden for giving me a constant outlet to find a healthy balance with sports.

Finally, and most importantly, I want to thank my friends and family for being there for me and reminding constantly of all the important and enjoyable aspects of life, besides academics.
To all my friends in and around Dresden, who accompanied me through life these past six years, thank you for making this one of the best times of my life. 
To my friends back in Aachen, San Salvador and spread throughout the rest of the world, thanks for being a constant source of love and friendship that has kept me grounded.
I won't list everyone who has made my life better these last six years and whom I consider a friend, I'm sure they know who they are, and I thank each and every one.

I would certainly not be who I am, and this thesis would not be possible, without the tremendous support from my family. 
My cousins, uncles and (great) aunts, my two big sisters, thank you everyone for always being there for me. 
Especially my little sister Ute, who's accompanied me a large part of my time here in Dresden, being a constant source of support and inspiration.
My father made me be curious and think critically since I was a kid, and coupled this inspiration with unconditional love, which I am certain was an indispensable for me to write this thesis.
My mother made me be social and empathic, and made sure I became a well-rounded person.
Her constant support and openness made me always do what interested me, and I am certain this thesis would never have happened without her.
Thank all of you for everything!