Having defined Reactors formally, we consider some applications for the model.
In this section we will discuss Reactors in the 5G standard.

Telecommunication standards evolve constantly, pushing the limits of signal processing systems from almost every angle.
Consumer demands adapt to increases in capabilities. This results in a feedback loop that not only raises the demands themselves, but also their heterogeneity.
In \ac{LTE} today we already see very dynamic demands, with different users requiring very different bandwidths at different times.
With the increased capabilities of 5G, the dynamicity of the demand will only increase.

Signal processing systems, however, are not built for dynamic workloads; they must tolerate the worst case.
This makes sense, since a system that is capable of processing the highest demands can also process lower demands.
However, parameters like user count, resource blocks supported, used MIMO scheme and carrier aggregation have a nuanced relationship in terms of resources pressure.
Additionally, the sub-carrier spacing is also flexible in 5G systems.
As a direct consequence, the real-time requirements have to adapt to the changing transmission time interval.
All of this yields a parameter space with a large dynamic range of possible workloads.

\begin{figure}[h]
	\centering
   \resizebox{0.95\textwidth}{!}{\usetikzlibrary{calc}
\usetikzlibrary{positioning}
\usetikzlibrary{shapes}
\usetikzlibrary{intersections,decorations.markings}
\usetikzlibrary{arrows.meta}
\usetikzlibrary{fit}
\usetikzlibrary{automata}

%\usepackage{../includes/nick-shorthands}

\newcommand{\lblTextsize}{\small}

%% COMMANDS %%

\newcommand{\textoverline}[1]{{$\overline{\mbox{#1}}$}}
\newcommand{\textinmath}[1]{\textnormal{\scriptsize{#1}}}

\newcommand{\pathdots}[2]{\path (#1) -- node[pos=0.54,sloped,font=\Huge]{\dots} (#2);}

\newcommand{\arrow}[2]{\draw [-latex] (#1) -- (#2);}
\newcommand{\arrowlbl}[4][]{\draw [-latex] (#2) -- node[#1]{#4} (#3);}
\newcommand{\edgearcarrowlbl}[5][]{\draw [-latex] (#2) -- ++(#4) -- node[#1]{#5} ($(#3)+(#4)$) -- (#3);}

%\newcommand{\algoloop}[1][2][{\node at ($(+#1.west)+(-0.5,+0.5)$) {frontend};
%	\draw [-latex, thick, rounded corners] ($(transmit.north west)+(+0.2,+0.2)$) -| ($(transmit.south east)+(+0.2,-0.2)$) -- ($(transmit.south west)+(-0.2,-0.2)$) |- ($(transmit.north west)+(+0.2,+0.2)$);


%% GENERIC %%

\usetikzlibrary{calc}

\tikzset{
	%
	box/.pic={
		\node [ draw, rectangle, minimum width=1.2cm, minimum height=1.2cm ] at (0,0) (-box) {};
	},
	myellipse/.pic={
		\node [ draw, ellipse, minimum width=2cm, minimum height=1cm ] at (0,0) (-box) {};
	}
	%
}
\tikzset{
%
greenbox/.pic={
	\node [ draw=green, very thick, rectangle, minimum width=1.2cm, minimum height=1.2cm ] at (0,0) (-box) {};
}
%
}

\tikzset{
	%
	shuffle-arrows/.pic={
		\begin{scope}[scale=0.6,shift={(-0.5,-0.25)}]
			\draw[->, thick, line cap=round] (0.0,0.5) .. controls (0.5,0.5) and (0.5,0.0) .. (1.0,0.0);
			\draw[line width=2.5pt, white] (0.0,0.0) .. controls (0.5,0.0) and (0.5,0.5) .. (1.0,0.5);
			\draw[->, thick, line cap=round] (0.0,0.0) .. controls (0.5,0.0) and (0.5,0.5) .. (1.0,0.5);
		\end{scope}
	}
	%
}

\tikzset{
	%
	pics/constellation/.style args={#1}{
		code={
			\draw[#1, line cap=round] (-0.33,0) -- (0.33,0);
			\draw[#1, line cap=round] (0,0.33) -- (0,-0.33);
			\draw[#1, fill=#1] (-0.15,-0.15) circle (0.04);
			\draw[#1, fill=#1] (0.15,-0.15) circle (0.04);
			\draw[#1, fill=#1] (-0.15,0.15) circle (0.04);
			\draw[#1, fill=#1] (0.15,0.15) circle (0.04);
			\draw [#1] (0,0) circle (0.212);
		}
	}
	%
}

\tikzset{
	%
	wave/.pic={
		\begin{scope}[scale=0.7, shift={(-0.6,-0.35)}]
			\begin{axis}[
				xmin=-70, xmax=790, 
				ymin=-1.1, ymax=1.1,
				hide x axis,
				hide y axis,
				width=8em,
				height=6.5em
			]
				\addplot [domain=0:720, samples=100, thick, line cap=round] {sin(x)};
			\end{axis}
		\end{scope}
	}
	%
}

\tikzset{
	%
	pics/smallwave/.style args={#1}{
		code={
		\begin{scope}[scale=0.7, shift={(-0.6,-0.35)}]
			\begin{axis}[
				xmin=-0.1, xmax=4.1, 
				ymin=-1.1, ymax=1.1,
				hide x axis,
				hide y axis,
				width=8em,
				height=6.5em,
				color=#1
			]
				\addplot [domain=0:4, samples=100, thick, line cap=round] {0.4*sin(x*360)};
			\end{axis}
		\end{scope}
	}
},
pics/smallwave/.default=black
}


\tikzset{
%
pics/shortwave/.style args={#1}{
	code={
	\begin{scope}[scale=0.7, shift={(-0.6,-0.35)}]
		\begin{axis}[
			xmin=-0.1, xmax=1.9, 
			ymin=-1.1, ymax=1.1,
			hide x axis,
			hide y axis,
			width=5.5em,
			height=6.5em,
			color=#1
			]
			\addplot [domain=0:4, samples=100, thick, line cap=round] {0.4*sin(x*360)};
		\end{axis}
	\end{scope}
	}
},
pics/shortwave/.default=black
}

\tikzset{
	%
	pics/sinc/.style args={#1}{
		code={
		\begin{scope}[scale=0.7, shift={(-0.6,-0.35)}]
			\begin{axis}[
				xmin=-700, xmax=700, 
				ymin=-0.5, ymax=2.1,
				hide x axis,
				hide y axis,
				width=8em,
				height=6.5em,
				color=#1
			]
				\addplot [domain=-680:680, samples=200, thick, line cap=round] {sin(x)/(x/360*3.14)};
			\end{axis}
		\end{scope}
	}
},
pics/sinc/.default=black
}

\tikzset{
%
cp_cs/.pic={
	\begin{scope}[scale=0.7, shift={(-0.6,-0.35)}]
		\begin{axis}[
			xmin=-700, xmax=700, 
			ymin=-2, ymax=2,
			hide x axis,
			hide y axis,
			width=8em,
			height=6.5em
			]
			\addplot [domain=-680:680, samples=200, thick, line cap=round] {cos(x)};
		\end{axis}
	\end{scope}
}
}

\tikzset{
	%
	whitebox/.pic={
		\node [ draw, rectangle, fill=white, minimum width=.05cm, minimum height=.5cm ] at (0,0) (-box) {};
	}
	%
}



\tikzset{
	%
	pics/rnd1/.style args={#1}{
		code={
		\begin{scope}[scale=0.7, shift={(-0.6,-0.35)}]
			\begin{axis}[
				xmin=-700, xmax=700, 
				ymin=-5, ymax=5,
				hide x axis,
				hide y axis,
				width=7em,
				height=6.5em,
				color=#1
				]
				\addplot [domain=-680:680, samples=200, thick, line cap=round] {sin(x)-3};
				\addplot [domain=-680:680, samples=200, thick, line cap=round] {cos(x)};
				\addplot [domain=-680:680, samples=200, thick, line cap=round] {sin(x)+3};
			\end{axis}
		\end{scope}
	}
},
pics/rnd1/.default=black
}

\tikzset{
	%
	die/.pic={
		\draw (-0.3,-0.3) rectangle (0.3,0.3);
	}
	%
}

\tikzset{
	%
	magnifier/.pic={
		\draw[thick, line cap=round, inner sep=0] (0,0) circle (0.14);
		\draw[very thick, line cap=round, inner sep=0] (0.11,-0.11) -- (0.22,-0.22);
	}
	%
}

\tikzset{
	%
	ruler/.pic={
		\begin{scope}[scale=0.045]
			\draw (-10,2) rectangle (10,-2);
			\foreach \x in {-9,-6,...,9} {%
				\draw (\x,2) -- (\x,0.2);
				\ifnum \x<9
					\foreach \y in {1,2} {%
						\draw ($(\x,2)+(\y,0)$) -- ($(\x,1)+(\y,0)$);
					};
				\fi
			};
		\end{scope}
	}
	%
}



%% COMPONENTS %%

\tikzset{
	interleaver/.pic={
		\pic[shift={(0,-0.2)}]{shuffle-arrows};
		\node at (0.00,0.3) {\textsf{Intv}};
		\pic {box};
	}
}

\tikzset{
	deinterleaver/.pic={
		\pic[shift={(0,-0.2)}]{shuffle-arrows};
		\node at (0.00,0.3) {\textsf{De-Intv}};
		\pic {box};
	}
}

\tikzset{
	enc-nsnr/.pic={
		\node at (0.0, 0.35) {\textsf{\small{Cv-Enc}}};
		\node at (0.0,-0.02) {\textsf{\large{\textoverline{S}-\textoverline{R}}}};
		\node at (0.0,-0.42) {\textsf{\footnotesize{1/2}}};
		\pic {box};
	}
}

\tikzset{
	enc-sr-acc/.pic={
		\node at (0.0, 0.35) {\textsf{\small{Cv-Enc}}};
		\node at (0.0,-0.02) {\textsf{\large{S-R}}};
		\node at (0.0,-0.38) {\textsf{\footnotesize{ACC}}};
		\pic {box};
	}
}

\tikzset{
	dec-nsnr/.pic={
		\node at (0.0, 0.35) {\textsf{\small{Cv-Dec}}};
		\node at (0.0,-0.02) {\textsf{\large{\textoverline{S}-\textoverline{R}}}};
		\node at (0.0,-0.42) {\textsf{\footnotesize{1/2}}};
		\pic {box};
	}
}

\tikzset{
	dec-sr-acc/.pic={
		\node at (0.0, 0.35) {\textsf{\small{Cv-Dec}}};
		\node at (0.0,-0.02) {\textsf{\large{S-R}}};
		\node at (0.0,-0.38) {\textsf{\footnotesize{ACC}}};
		\pic {box};
	}
}

\tikzset{
	encoder/.pic={
		\node at (0,-0.18) {%
			\footnotesize{$\overline{\underline{1010}}$}%
			\hspace{0.4ex}%
			\footnotesize{$\overline{\underline{11}}$}%
		};
		\node at (0.00,0.34) {\textsf{\small{Encod}}};
		\pic {box};
	}
}

\tikzset{
	pics/decoder/.style args={#1}{
		code={
			\node [#1] at (0,-0.18) {%
				\footnotesize{$\overline{\underline{1010}}$}%
				\hspace{0.4ex}%
				\footnotesize{$\overline{\underline{11}}$}%
			};
			\node [#1] at (0.00,0.34) {\textsf{\small{Decod}}};
		}
	},
	pics/decoder/.default=black
}

\tikzset{
	mapper/.pic={
		\pic[shift={(0,-0.18)}]{constellation};
		\node at (0.00,0.31) {\textsf{\small{Map}}};
		\pic {box};
	}
}

\tikzset{
	pics/demapper/.style args={#1}{
		code={
			\pic[shift={(0,-0.18)}]{constellation=#1};
			\node [#1] at (0.00,0.31) {\textsf{\small{Demap}}};
		}
	},
	pics/demapper/.default=black
}



\tikzset{
	resource_mapper/.pic={
		\pic[shift={(-0.08,-0.18)}]{shortwave};
		\pic[shift={(0.25,-0.18)}]{rnd1};
		\node at (0.00,0.34) {\textsf{\small{S/P}}};
		\pic {box};
	}
}

\tikzset{
	resource_mapper/.pic={
		\pic[shift={(-0.08,-0.18)}]{shortwave};
		\pic[shift={(0.25,-0.18)}]{rnd1};
		\node at (0.00,0.34) {\textsf{\small{S/P}}};
		\pic {box};
	}
}

\tikzset{
	pics/fft/.style args={#1}{
		code={
		\pic[shift={(-0.08,-0.18)}]{shortwave=#1};
		\pic[shift={(0.25,-0.18)}]{rnd1=#1};
		\node [#1] at (0.00,0.34) {\textsf{\small{FFT}}};
		}
	}
}

\tikzset{
	pics/idft/.style args={#1}{
		code={
		\pic[shift={(-0.08,-0.18)}]{rnd1=#1};
		\pic[shift={(0.6,-0.18)}]{shortwave=#1};
		\node [#1] at (0.00,0.34) {\textsf{\small{IDFT}}};
		}
	}
}



\tikzset{
	cp/.pic={
		\pic[shift={(0,-0.18)}]{cp_cs};
		\pic[shift={(-0.12,-0.17)}, scale=0.95, white] {whitebox};
		\node at (0.00,0.34) {\textsf{\small{CP/CS}}};
		\pic {box};
	}
}


\tikzset{
	modulator/.pic={
		\pic[shift={(0,-0.18)}]{sinc};
		\node at (0.00,0.34) {\textsf{\small{Mod}}};
		\pic {greenbox};
	}
}

\tikzset{
	demodulator/.pic={
		\pic[shift={(0,-0.18)}]{sinc};
		\node at (0.00,0.34) {\textsf{\small{Demod}}};
		\pic {greenbox};
	}
}

\tikzset{
	equalizer/.pic={
		\pic[shift={(0,-0.18)}]{sinc};
		\node at (0.00,0.34) {\textsf{\small{Inv}}};
	}
}

\tikzset{
	detector/.pic={
		\pic[shift={(0,-0.18)}] {smallwave};
		\pic[shift={(0,-0.17)}, scale=0.95, white] {magnifier};
		\pic[shift={(0,-0.17)}, scale=1.27, white] {magnifier};
		\pic[shift={(0,-0.17)}, scale=1.1] {magnifier};
		\node at (0.00,0.34) {\textsf{\small{Detect}}};
		\pic {box};
	}
}
\tikzset{
	synchronizer/.pic={
		\pic[shift={(0,-0.18)}] {smallwave};
		\pic[shift={(0,-0.17)}, scale=0.95, white] {magnifier};
		\pic[shift={(0,-0.17)}, scale=1.27, white] {magnifier};
		\pic[shift={(0,-0.17)}, scale=1.1] {magnifier};
		\node at (0.00,0.31) {\textsf{\small{Sync}}};
	}
}

\tikzset{
	estimator/.pic={
		\pic[shift={(0,-0.14)}, scale=0.55] {constellation};
		\draw[white, fill=white] (-0.4,-0.24) rectangle (0.4,-0.5);
		\pic[shift={(0,-0.33)}, scale=0.8] {ruler};
		\node at (0.00,0.34) {\textsf{\small{Estim}}};
	}
}

\tikzset{
	source/.pic={
		\node at (0,0.34) {\textsf{\small{Source}}};
		\node at (0,-0.18) {\small{$\overline{\underline{1010}}$}};
		\pic {box};
	}
}

\tikzset{
	data/.pic={
		\node at (0,0.34) {\textsf{\small{Data}}};
		\node at (0,-0.18) {\small{$\overline{\underline{1010}}$}};
		\pic {box};
	}
}

\tikzset{
	rayleigh/.pic={
		\begin{scope}[shift={(-0.43,-0.16)}]
			\begin{axis}
				[ width=69.5pt, height=62pt, axis x line=bottom, outer axis line style={-}, xtick=\empty, ytick=\empty ]
				\addplot [samples=200] { 1/sqrt(1-x^2) };
			\end{axis}
		\end{scope}
		\node at (0,-0.4) {\textsf{\scriptsize{Rayleigh}}};
		\pic {box};
	}
}

\tikzset{
	white-gaussian/.pic={
		\begin{scope}[shift={(-0.43,-0.16)}]
			\begin{axis}
				[ width=69.5pt, height=62pt, axis x line=bottom, outer axis line style={-}, xtick=\empty, ytick=\empty ]
				\addplot [samples=200] { 1/sqrt(6*3^2) * exp( -x^2 / sqrt(6*3^2) ) };
			\end{axis}
		\end{scope}
		\node at (0,-0.4) {\textsf{\scriptsize{Gauss}}};
		\pic {box};
	}
}

\tikzset{
	hard-decision/.pic={
		\node at (0,0.28) {\textsf{\small{Hard}}};
		\draw[thick,shift={(0,-0.2)}] (-0.2,-0.2) -- (0,-0.2) -- (0,0.2) -- (0.2,0.2);
		\pic {box};
	}
}

\tikzset{
	add/.pic={
		\draw [ line cap=round ] (-0.2,0) -- (0.2,0);
		\draw [ line cap=round ] (0,-0.2) -- (0,0.2);
		\draw (0,0) circle (0.25);
		\node [ minimum width=0.5cm, minimum height=0.5cm ] at (0,0) (-box) {};
	}
}

\tikzset{
	multiply/.pic={
		\draw [ line cap=round ] (-0.141,-0.141) -- (+0.141,+0.141);
		\draw [ line cap=round ] (-0.141,+0.141) -- (+0.141,-0.141);
		\draw (0,0) circle (0.25);
		\node [ minimum width=0.5cm, minimum height=0.5cm ] at (0,0) (-box) {};
	}
}
\tikzset{
	blub/.style args={#1}{draw=#1, inner sep=0, ellipse, fill=white, minimum width=1.2cm, minimum height=0.5cm},
	copy/.style={blub, color=gray!80},
	pics/antenna/.style args={#1}{
		code={
			\draw [color=#1] (0,0) -- (50:0.5);
			\draw [color=#1] (0,0) -- (130:0.5);
			\draw [color=#1] (0,0) -- (0,-0.5);
		}
	},
	pics/frontend/.style args={#1}{
		code={
			\node[inner sep=0] (-ant) {
				\tikz[xshift=2cm]{ 
					\pic [anchor=west] at (0,0) {antenna=#1};
				}
			};	
%			\pic (-ant) at (0,0) {antenna=#1};
			
			\node[blub=#1, right= of -ant] (-b) {
				\tikz{ 
					\pic [anchor=center]{synchronizer};
				}
			};
			\node[blub=#1, right= of -b] (-c) {
				\tikz{ 
					\pic [anchor=center] at (0,0) {fft=#1};
				}
			};
			\node[blub=gray!80, below= of -c] (-d) {
				\tikz{ 
					\pic [anchor=center] at (0,0) {fft=gray!80};
				}
			};
			\node[blub=#1, right= of -c] (-e) {
				\tikz{ 
					\pic [anchor=center] at (0,0) {estimator};
				}
			};

			\draw [-Stealth, color=#1] (-ant) -- (-b);
			\draw [-Stealth, color=#1, name path=-t] (-b) -- (-c);
			\draw [-Stealth, dashed, color=gray!80] (-b) -- (-d);	
			\draw [-Stealth, color=#1] (-c) -- (-e);	
		}
	},
	pics/backend/.style args={#1}{
		code={
			\node[blub=#1] (-a) {
				\tikz{ 
					\pic [inner sep=0, anchor=center] at (0,0) {idft=#1};
				}
			};
			
			\node[blub=#1, right= of -a] (-b) {
				\tikz{ 
					\pic [inner sep=0, anchor=center] at (0,0) {demapper=#1};
				}
			};
			
			\node[blub=#1, right= of -b] (-c) {
				\tikz{ 
					\pic [inner sep=0, anchor=center] at (0,0) {decoder=#1};
				}
			};
		
			\draw [-Stealth, color=#1] (-a) -- (-b);
			\draw [-Stealth, color=#1] (-b) -- (-c);
		}
	}
}


\begin{tikzpicture}

	\draw [yshift=0] pic (A3) {frontend=gray!80};
	\draw [yshift=3] pic (A2) {frontend=gray!80};
	\draw [yshift=6] pic (A1) {frontend=gray!80};
	\draw [yshift=9]  pic (A0) {frontend=black};
	
	
	\node at (1,-1) (antLabel) {\lblTextsize\#Antennas};
	\node at (2,-2) (carrierLabel) {\lblTextsize\#Carriers};
	\node at ($(A0-c.south)!0.6!(A0-d.north)$) (scCnt) {\lblTextsize\#Sub-Carriers};
	
	\draw [->, dashed] (antLabel)  -- ($(A3-ant.east)!0.5!(A3-b.west)$);
	\draw [->, dashed] (carrierLabel.north)  -- ($(A0-b.east)!0.5!(A3-c.west)$);
	\draw [->, dashed] (carrierLabel.north)  -- ($(A0-b.south east)!0.4!(A3-d.north west)$);
	\draw [->, dashed] (scCnt.north)  -- ++(90:0.5cm);
	
	\node[blub=black, below= of A0-e] (B) {
		\tikz{ 
			\pic [anchor=center] at (0,0) {equalizer};
		}
	};

	\foreach \y in {4,3,...,1}{
		\pgfmathparse{\y*3}
		\edef\yStart{\pgfmathresult}
		\node[yshift=\yStart, right= of B, inner sep=0] (C\y) {
			\tikz{ 
				\pic [inner sep=0, anchor=center] at (0,0) {backend=gray!80};
			}
		};
		\draw [gray!80, -Stealth] (B.east) -- (C\y.west);
	}

	\node[right= of B, inner sep=0] (C) {
		\tikz{ 
			\pic [inner sep=0, anchor=center] {backend=black};
		}
	};

	\foreach \y in {3,2,1}{
		\pgfmathparse{\y*3}
		\edef\yStart{\pgfmathresult}
%		\node[yshift=\yStart, above= of C, align=center, inner sep=0] (C_\y) {
%			\tikz{ 
%				\pic [inner sep=0, anchor=center] at (0,0) {backend=gray!80};
%			}
%		};
	
		\draw  pic [anchor=west] (C_\y) at ([yshift=\yStart] C.west |- A0-e.east) {backend=gray!80};
		\draw [gray!80, -Stealth] (B.north east) -- (C_\y-a.south west);
	}


	\draw  pic [anchor=west] (C_0) at (C.west |- A0-e.east) {backend=gray!80};
	\draw [gray!80, -Stealth] (B.north east) -- (C_0-a.south west);
	
	\node at ([yshift=-0.35cm]C_0-a.south) (layerCnt) {\lblTextsize\#Layers};
	\node at ([yshift=-0.35cm]C_0-b.south) (rbCnt) {\lblTextsize\#RBs};
	\node at ([yshift=-0.35cm]C_0-c.south) (userCnt) {\lblTextsize\#Users};
	
	\draw [->, dashed] (rbCnt.west) -- ++(210:2.8cm);
	\draw [->, dashed] (userCnt.west) -- ++(235:1.4cm);
	\draw [->, dashed] (layerCnt.west) -- ++(195:0.6cm);
	\draw [->, dashed] (layerCnt.west) -- ++(245:1cm);


	\draw [-Stealth] (A0-e) -- (B);
	\draw [-Stealth, gray!80] (A1-c) -- (B);
	\draw [-Stealth, gray!80] (A2-c) -- (B);
	\draw [-Stealth, gray!80] (A3-c) -- (B);
	\draw [-Stealth] (A0-c) -- (B);
	
	\draw [-Stealth] (B) -- (C);
	
	\draw  ($ (A0-b.south east)!0.4!(A0-c.south west) $ ) arc(-120:120:0.2 and 0.35);
	\draw  [rotate=-45] ([shift=(-90:0.4cm)]$ (A0-b.south east)!0.4!(A0-d.north west) $ ) arc(-120:120:0.2 and 0.35);

	\draw  ([yshift=-2.5]$ (B.east)!0.4!(C.west) $ ) arc(-150:150:0.2 and 0.35);
	\draw  [rotate=45] ([shift=(90:-0.1cm)]$ (B.north east)!0.4!(C_0-a.south west) $ ) arc(-120:120:0.1 and 0.15);


%	\coordinate (origin) at (0,0);
%	\coordinate (xmax) at ([xshift=8.5cm] origin);
%	\coordinate (ymax) at ([yshift=3.8cm] origin);
%	\coordinate (y1) at ([yshift=2.5cm] origin);
%	
%	\foreach \x/\xl in {0/, 1/0.1M, 2/1M/, 3/10M, 4/100M, 5/1G, 6/10G, 7/100G, 8/1T}
%	\draw (\x,0.0) -- (\x,-0.1) node [below] {\xl};
%	\foreach \y/\yl in {0/1000, 1/100, 2/10, 3/1}
%	\draw (0.0,\y) -- (-0.1,\y) node [left] {\yl};
%	
%	\draw [->, very thick] (origin) -- (xmax) node [midway, below, yshift=-15](xlabel) {Throughput [bps]};
%	\draw [->, very thick] (origin) -- (ymax) node [midway, left, xshift=-20, rotate=90, anchor=south](ylabel) {Latency [ms]};
%
%	\draw [rounded corners, fill=gray!5, name path=6G] (0,0) rectangle (8,3.2) node[below left=]{6G};
%	\draw [rounded corners, fill=gray!10, name path=5G] (0,0) rectangle (6,2.4) node[below left=]{5G};
%	\draw [rounded corners, fill=gray!15, name path=4G] (0,0) rectangle (4.5,1.3) node[below left=]{4G};
%	\draw [rounded corners, fill=gray!80, name path=3G] (0,0) rectangle (3.1,1) node[below left=]{3G};
%	\draw [rounded corners, fill=gray!25, name path=2G] (0,0) rectangle (1.3,1) node[below left=]{2G};
%	
%	\draw [Stealth-Stealth, red, ultra thick] (0.3,0.25cm) -- ++ (7.7cm,0)
%		node [very near end, above]{$\sim 10^7$};
%	\draw [Stealth-Stealth, red, ultra thick] (0.25,0.3cm) -- ++ (0, 2.9cm)
%	node [very near end, right]{$\sim 10^3$};
%	
	
\end{tikzpicture}}
   \caption{Simplified model of a basestation uplink  modem. Adapted from Figure~2 of~\cite{wittig_ict20}} 
	\label{fig:5g_flow}
\end{figure}

Figure~\ref{fig:5g_flow} shows a simplified overview of the uplink modem in a basestation for 5G. 
We see that the overview already resembles \acp{MoC} like dataflow or Reactors.
Details on the requirements, like the sizes and numbers of \ac{FFT} nodes depend significantly on the workload being processed.
However, the dependencies between the resources required for the baseband processing and the parameters of the workload are non-trivial.
Figure~\ref{fig:5g_radar} shows a small selection of parameter combinations for \ac{LTE} baseband processing and how the required computations depend on them.
In the figure, the number of antennas and layers are both fixed at $1$. These also affect the required numbers of \ac{FFT}, \ac{IFFT}, equalization, demodulation and channel inversion nodes.
The length of the values in the radar plot shows the relative amount of kernels required for each type of operation.

\begin{figure}[t]
	\centering
	%\includegraphics[width=0.46\textwidth]{figures/statistics.pdf}
	\resizebox{0.95\textwidth}{!}{\inputTikz{5g_radar.tex}}
	\caption{Different parameter combinations and their effects on the requirements on computation in \ac{LTE}.}
	\label{fig:5g_radar}
\end{figure}
        
%The adaptability of a system does not come for free.
Even for fixed parameters, designing modems for 5G is already an extremely complex endeavour on its own. 
In order to make a system adaptable, developers need to ensure that the behavior remains correct when adapting.
A model that ensures deterministic execution for easy debugging and time semantics for real-time reasoning is thus sought for this type of applications. 
%Determinism in the execution is an invaluable tool for this.
%Due to the real-time nature of the problem, having a timed model is also very useful.
For this reason, we propose to use the Reactors model~\cite{wittig_ict20}.
We adapted the WiBench benchmark~\cite{wibench} (cf. Figure~\ref{fig:reactors_wibench}), dealing with \acs{PHY} in \ac{LTE}, using Lingua Franca~\cite{lingua_franca}.
Lingua Franca is an implementation of the Reactors model.

\subsubsection{Adaptability in 5G and beyond}

We used LTE traces to extrapolate information about the dynamicity of the demands for 5G and beyond.
These traffic traces, collected over a $5$ hour period spread over $15$ days feature real data with over $1.2$ million \acp{RNTI} from $24$ different base stations~\cite{budhdev2020isoran}.
They were generously shared by Arka Maity, Nishant Budhev and Tulika Mitra.

Figure~\ref{fig:statistics} shows statistical data points extracted from the traces.
In the figure, every point represents the workload of a base station at a particular set of subframes.
We consider the relationship between the number of \ac{UE} units and the total number of resource blocks required at those subframes.
The size of each point represents how many subframes in the trace had those precise requirements in terms of \acp{UE} and total resource blocks.
%We also classify three types of design points, as follows:

\begin{figure}[t]
	\centering
	%\includegraphics[width=0.46\textwidth]{figures/statistics.pdf}
	\resizebox{0.95\textwidth}{!}{\inputTikz{lte_statistics.tex}}
	\caption{Possible configurations in a resource-constrained \acs{LTE} environment. The number of UEs are depicted with a meaningless random jitter for visibility. Adapted from Figure~2 in~\cite{wittig_ict20}.}
	\label{fig:statistics}
\end{figure}

In a traditional setup, with a static implementation of a \acs{PHY}, we would make worst-case assumptions.
This includes the number of \acp{PE} and resource blocks per \ac{UE} that we can support, as dictated by resource constraints.
Let's assume that, given our limited resources, our \acs{PHY} implementation supports at most $10$ \acp{UE} and $47$ resource blocks per \ac{UE}.
These numbers are admittedly low, even for \acs{LTE}.
However, we choose this low threshold deliberately.
This way we can use these \ac{LTE} traces to extrapolate the possible dynamicity of behaviors in 5G and beyond.
Out of the $3017424$ considered subframes in the traces, slightly over half ($1689447$) could be processed with these static resource constraints.
These design points we classify as ``Static'' and depict as green points in Figure~\ref{fig:statistics}.
Consider a dynamic system that can adapt to the current workload, e.g. by supporting less \acp{UE} but more resource blocks per \ac{UE}, or vice-versa.
We can estimate the resources required for different configurations by generating task graphs for each and comparing the computational resources required.
Some design points in Figure~\ref{fig:statistics} can be supported by a dynamic system using the same resources as in the static version.
We classify them as ``Dynamic''.
Slightly less than half of the subframes in the trace ($1327900$) fall under this category.
Finally, the remaining $77$ points would need more resources to be supported, even with a dynamic system.
We classify them as ``Impossible'' (red points).
Overall, this means that over $99.997\%$ of the subframes observed could be implemented by a dynamic system, where a static one using roughly the same resources covers less than $56\%$.

This statistical analysis of LTE traces shows very clearly how much a baseband system could benefit from dynamically adapting to the workload.
We can cover significantly more design points with constrained resources. Equivalently, we can use less resources to cover all of the observed cases.
However, such dynamic systems come with several drawbacks.
Programming base station decoders is already a complex endeavor for static systems, much more so with dynamic ones.
With real-time requirements, we must ensure that the changing system not only respects the deterministic semantics of the decoder, but also the timing requirements.
This is why we propose to use a formal model of computation to describe 5G (and beyond). Using the model of Reactors we can make the execution deterministic and timed.
It also can help define well-behaved dynamic behavior through the use of mutations in future work.

\subsubsection{Modeling 5G with Reactors}

\begin{figure}[t]
	\centering
	\resizebox{1\textwidth}{!}{%% Creator: Inkscape inkscape 0.91, www.inkscape.org
%% PDF/EPS/PS + LaTeX output extension by Johan Engelen, 2010
%% Accompanies image file 'figures/reactors_wibench.pdf' (pdf, eps, ps)
%%
%% To include the image in your LaTeX document, write
%%   \input{<filename>.pdf_tex}
%%  instead of
%%   \includegraphics{<filename>.pdf}
%% To scale the image, write
%%   \def\svgwidth{<desired width>}
%%   \input{<filename>.pdf_tex}
%%  instead of
%%   \includegraphics[width=<desired width>]{<filename>.pdf}
%%
%% Images with a different path to the parent latex file can
%% be accessed with the `import' package (which may need to be
%% installed) using
%%   \usepackage{import}
%% in the preamble, and then including the image with
%%   \import{<path to file>}{<filename>.pdf_tex}
%% Alternatively, one can specify
%%   \graphicspath{{<path to file>/}}
%% 
%% For more information, please see info/svg-inkscape on CTAN:
%%   http://tug.ctan.org/tex-archive/info/svg-inkscape
%%
\begingroup%
  \makeatletter%
  \providecommand\color[2][]{%
    \errmessage{(Inkscape) Color is used for the text in Inkscape, but the package 'color.sty' is not loaded}%
    \renewcommand\color[2][]{}%
  }%
  \providecommand\transparent[1]{%
    \errmessage{(Inkscape) Transparency is used (non-zero) for the text in Inkscape, but the package 'transparent.sty' is not loaded}%
    \renewcommand\transparent[1]{}%
  }%
  \providecommand\rotatebox[2]{#2}%
  \ifx\svgwidth\undefined%
    \setlength{\unitlength}{563.89536526bp}%
    \ifx\svgscale\undefined%
      \relax%
    \else%
      \setlength{\unitlength}{\unitlength * \real{\svgscale}}%
    \fi%
  \else%
    \setlength{\unitlength}{\svgwidth}%
  \fi%
  \global\let\svgwidth\undefined%
  \global\let\svgscale\undefined%
  \makeatother%
  \begin{picture}(1,0.65267091)%
    \put(0,0){\includegraphics[width=\unitlength,page=1]{figures/reactors_wibench.pdf}}%
    \put(0.6526984,0.08228477){\color[rgb]{0,0,0}\makebox(0,0)[lb]{\smash{WiBench}}}%
    \put(0,0){\includegraphics[width=\unitlength,page=2]{figures/reactors_wibench.pdf}}%
    \put(0.01844314,0.6441587){\color[rgb]{0,0,0}\makebox(0,0)[lb]{\smash{GenerateInputs}}}%
    \put(0,0){\includegraphics[width=\unitlength,page=3]{figures/reactors_wibench.pdf}}%
    \put(0.00851222,0.51221933){\color[rgb]{0,0,0}\makebox(0,0)[lb]{\smash{(0, 1nsec)}}}%
    \put(0,0){\includegraphics[width=\unitlength,page=4]{figures/reactors_wibench.pdf}}%
    \put(0.13193937,0.55265236){\color[rgb]{0,0,0}\makebox(0,0)[lb]{\smash{\textbf{2}}}}%
    \put(0,0){\includegraphics[width=\unitlength,page=5]{figures/reactors_wibench.pdf}}%
    \put(0.13193937,0.60798177){\color[rgb]{0,0,0}\makebox(0,0)[lb]{\smash{\textbf{1}}}}%
    \put(0,0){\includegraphics[width=\unitlength,page=6]{figures/reactors_wibench.pdf}}%
    \put(0.23834209,0.58882928){\color[rgb]{0,0,0}\makebox(0,0)[lb]{\smash{Encoder}}}%
    \put(0,0){\includegraphics[width=\unitlength,page=7]{figures/reactors_wibench.pdf}}%
    \put(0.37892474,0.58882928){\color[rgb]{0,0,0}\makebox(0,0)[lb]{\smash{RateMatcher}}}%
    \put(0,0){\includegraphics[width=\unitlength,page=8]{figures/reactors_wibench.pdf}}%
    \put(0.55976562,0.58882928){\color[rgb]{0,0,0}\makebox(0,0)[lb]{\smash{Scrambler}}}%
    \put(0,0){\includegraphics[width=\unitlength,page=9]{figures/reactors_wibench.pdf}}%
    \put(0.71919299,0.58882928){\color[rgb]{0,0,0}\makebox(0,0)[lb]{\smash{Modulator}}}%
    \put(0,0){\includegraphics[width=\unitlength,page=10]{figures/reactors_wibench.pdf}}%
    \put(0.87927356,0.58845893){\color[rgb]{0,0,0}\makebox(0,0)[lb]{\smash{Precoder}}}%
    \put(0,0){\includegraphics[width=\unitlength,page=11]{figures/reactors_wibench.pdf}}%
    \put(0.04585149,0.41390658){\color[rgb]{0,0,0}\makebox(0,0)[lb]{\smash{SubCarrierMapper}}}%
    \put(0,0){\includegraphics[width=\unitlength,page=12]{figures/reactors_wibench.pdf}}%
    \put(0.26561509,0.41390658){\color[rgb]{0,0,0}\makebox(0,0)[lb]{\smash{SCFDMAModulator}}}%
    \put(0,0){\includegraphics[width=\unitlength,page=13]{figures/reactors_wibench.pdf}}%
    \put(0.5015007,0.41390658){\color[rgb]{0,0,0}\makebox(0,0)[lb]{\smash{ChannelReactor}}}%
    \put(0,0){\includegraphics[width=\unitlength,page=14]{figures/reactors_wibench.pdf}}%
    \put(0.70909382,0.41336561){\color[rgb]{0,0,0}\makebox(0,0)[lb]{\smash{SCFDMADemodulator}}}%
    \put(0,0){\includegraphics[width=\unitlength,page=15]{figures/reactors_wibench.pdf}}%
    \put(0.03746624,0.23781008){\color[rgb]{0,0,0}\makebox(0,0)[lb]{\smash{SubCarrierDemapper}}}%
    \put(0,0){\includegraphics[width=\unitlength,page=16]{figures/reactors_wibench.pdf}}%
    \put(0.28833335,0.23781008){\color[rgb]{0,0,0}\makebox(0,0)[lb]{\smash{EqualizerReactor}}}%
    \put(0,0){\includegraphics[width=\unitlength,page=17]{figures/reactors_wibench.pdf}}%
    \put(0.50868771,0.23781008){\color[rgb]{0,0,0}\makebox(0,0)[lb]{\smash{TransformDecoderReactor}}}%
    \put(0,0){\includegraphics[width=\unitlength,page=18]{figures/reactors_wibench.pdf}}%
    \put(0.80658014,0.23781008){\color[rgb]{0,0,0}\makebox(0,0)[lb]{\smash{Demodulator}}}%
    \put(0,0){\includegraphics[width=\unitlength,page=19]{figures/reactors_wibench.pdf}}%
    \put(0.03547694,0.08228477){\color[rgb]{0,0,0}\makebox(0,0)[lb]{\smash{Descrambler}}}%
    \put(0,0){\includegraphics[width=\unitlength,page=20]{figures/reactors_wibench.pdf}}%
    \put(0.23559506,0.08228477){\color[rgb]{0,0,0}\makebox(0,0)[lb]{\smash{RxRateMatcher}}}%
    \put(0,0){\includegraphics[width=\unitlength,page=21]{figures/reactors_wibench.pdf}}%
    \put(0.45130415,0.08228477){\color[rgb]{0,0,0}\makebox(0,0)[lb]{\smash{TurboDecoder}}}%
    \put(0,0){\includegraphics[width=\unitlength,page=22]{figures/reactors_wibench.pdf}}%
  \end{picture}%
\endgroup%
}
	%\includegraphics[width=0.46\textwidth]{figures/reactors_wibench.pdf}
	\caption{The Reactor network of the modified WiBench benchmark in Lingua Franca.}
	\label{fig:reactors_wibench}
\end{figure}


In ongoing (unpublished) work with Robert Wittig and Christian Menard, we adapted the WiBench benchmark~\cite{wibench} to work with Lingua Franca, an implementation of Reactors. 
Figure~\ref{fig:reactors_wibench} depicts the Reactor network implementing this benchmark. 
Since WiBench is single threaded, we only compared to a single threaded version in Reactors.
In particular, we did not leverage data level parallelism throughout the layer, nor the pipeline parallelism that we get from the network's topology for free.
This is a worst-case assumption we made to analyze the overhead.
By using the Reactor model, the benchmark is deterministic, even if it was to run using this parallelism~\cite{lohstroh_cyphy19}.
More importantly though, we can use the model's time semantics to define the constraints that ensure each subframe is processed on time.
Our implementation is thus still static (cf. Figure~\ref{fig:reactors_wibench}), since we have not yet specified well-defined mutations.
This implementation presents a great opportunity for future work to research and develop safe mutations for 5G.

Our implementation of the Reactor-based WiBench had an overhead of $15\%$ (median over $100$ executions), compared to the baseline implementation of WiBench.
There is certainly potential to improve this, e.g. as the scheduler of the C++ implementation on Lingua Franca, used for this implementation, was not optimized at all.
Nevertheless, this is a purely software-based implementation, so it serves only as a very rough estimation of the overhead; it is best suited to study the model's suitability and develop Reactor mutations for adaptability.
An efficient implementation in practice could work with reconfigurable hardware, e.g. implementing a \ac{PRET}~\cite{pret} machine, which is well-suited to Reactors' semantics.

In general, these preliminary results open up many avenues for research in adaptability in 5G. We can the Reactors model, at the semantic level to support the necessary adaptability in 5G. Similarly, we can design reconfigurable hardware that implements it.

\subsubsection{Other applications: Automotive}

The reactors model has many desirable properties for designing reliable \acp{CPS}, which can be applied in a multitude of domains.
An important example is the automotive domain, where the high-performance requirements of autonomous driving and modern entertainment are coupled with the timed \ac{CPS} including the car and its surroundings.
To keep the scope of this thesis limited, we omit a thorough discussion of an application of Reactors in the automotive domain.
In~\cite{menard_date20}, we showed how we can use the Reactors model to achieve determinism in the AUTOSAR \ac{AP}, a modern automotive standard.
