The first part of this thesis deals with dataflow-based software synthesis.
Software synthesis refers to a series of related methods, with the common goal of efficiently deploying software on a specific hardware architecture.
While the aim of this thesis is to analyze structures and present methods that are general to software synthesis as a whole, in practice we will always work with concrete design flows.
The first part of the thesis thus deals with design flows based around mapping dataflow applications to heterogeneous \acp{MPSoC}, on the example of a flow based on \acp{KPN}.
Many of the methods apply more generally to other models of computation.
For this flow we will investigate multiple structural properties of the applications, architectures and mappings and show how to exploit these.
Symmetries of applications and architectures result in symmetries of mappings, which can be exploited both for \ac{DSE} and at runtime.
The spatial structure of the mappings and the way mappings are represented can also be leveraged in different mapping algorithms, like those based bio-inspired design centering and genetic algorithms.
Finally, we discuss the structure of data allocation in memory.