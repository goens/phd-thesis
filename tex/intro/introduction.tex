Programming computers is notoriously difficult.
Indeed, people learning to program usually struggle, paradoxically, with the fact that the computer does precisely what they tell it to do. 
This is confusing, not because a computer program is executed faithfully, but rather, because humans think at a very different level of abstraction.

It is certainly true that instructions in computer architectures are at a completely different level of abstraction than the instructions we give each other.
However, most programs are also not written at the level of the architecture.
Programming languages are designed with increasingly improving abstractions, making it easier for programmers to express themselves.
Complementary to these efforts are compilers, which serve as bridge between the levels of abstraction.
Ideally, a compiler translates the abstract human-level expressions into efficient machine-level instructions.
While we have made significant progress, this task has proven to be dauntingly difficult.

Traditionally, we have put the research and effort into optimizing the execution of a single core.
Most of the progress of decades of research in programming language and compilers revolves around this single-core model.
In the last decade or two, however, with the multicore era, this challenge has increased dramatically.
Now we have to use and coordinate multiple cores, commonly with different capabilities.
The widespread programming language abstractions and compiler analyses of today are ill-suited to tackle this challenge.

There is probably no universal solution to these emerging problems, as different domains have different requirements.
This thesis thus focuses mostly a particular domain, that of \acfp{CPS} or generally, embedded systems.
In this domain, a family of methods called software synthesis seeks to enable efficient programming of complex multicore systems.
Central to these methods is a focus on using models for describing computation.
We follow the idea of letting theory inform practice, in striving to improve methods of software synthesis.
We do this by identifying and exploiting mathematical structures in the problems in software synthesis.

\section{The Multicore Era}

On the hardware side, the last two decades have firmly established what we call the multicore era.
Modern computing systems are almost universally composed of multiple logical cores, and there is a clear trend of increasing both the number and the degree of heterogeneity of these cores.
This increasing complexity brings about an increasing challenge in taming it.

\begin{figure}[h]
	\centering
   \resizebox{0.9\textwidth}{!}{\input{generated/moore.tex}}
   \caption{Chip trends as obtained from~\cite{microprocessordata}. The lines present the exponential growth prediction if considering data up until the year 2000.}
	\label{fig:multicore_era}
\end{figure}


%Multicore Era
Both the execution frequency and the closely intertwined single-core processing speed of computing systems increased exponentially up until the early 2000s (cf. Figure~\ref{fig:multicore_era}), an empirical fact observed by Gordon Moore in 1965~\cite{moore}.
Since the early 2000s, however, while transistor sizes continue to decrease, the exponential frequency scaling has stopped (cf. Figure~\ref{fig:multicore_era}).
Individual and clever designs have continued to improve single-core performance, albeit at a significantly slower rate.
Instead, most improvements in microchips in raw processing power have come from a paradigm switch.
Hardware designers resorted to develop multicores, microchips composed of multiple cores that can execute in parallel.
Additionally, since different use-cases benefit from different computing core architectures, the inclusion of multiple chips has paved the way for heterogeneity.
This has resulted in a multicore era of computing.

A programmer writing a piece of C code in the early 1970s would automatically benefit from the increasing processing speeds.
By recompiling her code for a processor $10 \times$ faster, she could very roughly expect her code also to run about $10 \times$ as fast. % Do I want to make this concrete (with Processor names?)
This has changed. A programmer writing a piece of (sequential) C code in the early 2000s cannot expect anywhere near a $100$-fold increase in performance today when she uses a microchip with $100 \times$ the number of transistors.
At least not without fundamentally restructuring her code to exploit parallelism.
Contemporary systems aimed at performance almost ubiquitously consist of multicore chips, in many cases heterogeneous, which tremendously increases the system complexity.
It is a clear trend that the complexity and heterogeneity will continue to increase~\cite{voelp16_pmes,castrillon_tmscs17}.
%The trend is clear, the number of cores, heterogeneity and overall system complexity will only continue to increase.

%something about network-on-chip, emerging memories, and hierarchical topologies
As the number of cores in a chip keeps increasing, the importance of coordination and communication between the cores raises as well.
Memory latency and bandwidth has been a bottleneck for many classes of applications for a while.
In the case of manycores, which are multicores where the number of cores goes over several dozens, even to hundreds or thousands, the on-chip memory subsystem becomes a central design point of the chip.
Systems based on \ac{NoC} technology become necessary, lest a single bus interconnect becomes the bottleneck of the system when thousands of cores want to communicate simultaneously through it.


\begin{figure}[h]
	\centering
   \resizebox{0.85\textwidth}{!}{%Based on design by Marco Miani
%https://texample.net/media/tikz/examples/TEX/swan-wave-model.tex

\begin{tikzpicture}[scale=.9,every node/.style={minimum size=1cm},on grid]

    %slanting: production of a set of n 'laminae' to be piled up. N=number of grids.
    \begin{scope}[
            yshift=-83,every node/.append style={
            yslant=0.5,xslant=-1},yslant=0.5,xslant=-1
            ]
        \fill[white,fill opacity=0.6] (0,0) rectangle (5,5);
        \draw[black,dashed] (0,0) rectangle (5,5);%marking borders
        \pic [opacity=0.3] {fourmesh};
    \end{scope}
    	
    \begin{scope}[
    	yshift=0,every node/.append style={
    	    yslant=0.5,xslant=-1},yslant=0.5,xslant=-1
    	             ]
        \fill[white,fill opacity=0.6] (0,0) rectangle (5,5);
        \draw[black,very thick] (0,0) rectangle (5,5);%marking borders
        \pic {fourmesh};
    \end{scope}
    	
    \begin{scope}[
    	yshift=90,every node/.append style={
    	yslant=0.5,xslant=-1},yslant=0.5,xslant=-1
    	             ]
    	\fill[white,fill opacity=.9] (0,0) rectangle (5,5);
    	\draw[black,dashed] (0,0) rectangle (5,5);
        \pic [opacity=0.3] {fourmesh};
    \end{scope}
    	
    \begin{scope}[
    	yshift=170,every node/.append style={
    	    yslant=0.5,xslant=-1},yslant=0.5,xslant=-1
    	  ]
        \fill[white,fill opacity=0.6] (0,0) rectangle (5,5);
        \draw[black,dashed] (0,0) rectangle (5,5);
        \pic [opacity=0.3] {fourmesh};
    \end{scope}

    %putting arrows and labels:
    \draw[-latex,thick] (6.2,2.8) node[right]{Manycore}
         to[out=180,in=90] (3.5,2.3);

    \draw[-latex,thick](5.8,.3)node[right, align=center]{Optical \\ Interconnect}
        to[out=180,in=90] (3.3,-0.8);

    \draw[-latex,thick](5.9,7.5)node[right,align=center]{Wireless \\inter-board \\ links}
        to[out=180,in=0] (3.6,7.0);
\end{tikzpicture}}
   \caption{The \acs*{HAEC} architecture~\cite{HAEC} has multiple levels of hierarchy: on-chip, intra-board (optical links) and inter-board (wireless).} 
	\label{fig:haec}
\end{figure}

In fact, for a multitude of reasons, manycores are commonly designed in a hierarchical fashion.
Smaller clusters of cores locally interconnected communicate between each other and off-chip via a larger \ac{NoC}.
These clusters and systems are usually heterogeneous as well, like the Karlay MPPA3 Coolidge~\cite{coolidge}, with includes accelerators for cryptography and secure cores, alongside general purpose cores.
Some systems even propose multiple layers of hierarchy, like the \ac{HAEC} topology~\cite{HAEC}, depicted in Figure~\ref{fig:haec}.
This is a proposed 3D-stacked system with multiple \acp{PCB} each with multiple manycore chips, not a single chip with this complex interconnect.
However, this design could allow for very low latencies, such that challenges of programming it are comparable to those of programming such a topology as part of an on-chip memory subsystem.
What happens if the mulicores in this system are similar to the Karlay MPPA3 Coolidge? This would yield more than $5000$ heterogeneous cores connected at five levels of hierarchy in different topologies.
More generally, complex topologies, with possibly multi-level hierarchies, add a layer of complexity to modern manycore systems besides heterogeneity and concurrency.
The techniques used for programming a system with two or four identical cores are mostly not suited at all to program these dauntingly complex topologies.
We need to re-think our design process to accomodate for these changes in complexity.
%\todo{JC: great segway to complexity of topologies: be more explicit! (not sure about what?)}

%As part of the shift towards memory-centered systems, emerging memory technologies play a significant role.
%In general, upcoming technologies like \acp{NVM} where data is not lost when power is turned off, have significant implications e.g. for energy efficient systems~\cite{meena2014nvm,mittal2014survey}\index{non-volatile memory}.
%Some concrete \acp{NVM}, however, have further unique implications that increase the complexity of the system.
%For example, in \acp{RTM} information is stored in a so-called racetrack that can store multiple data addresses~\cite{rtm}\index{racetrack memory}.
%An address can only be accessed for reading or writing by shifting the head of a reader to the correct location, which has costs in terms of energy and latency.
%Thus, considerations of memory layout change when targeting \acp{RTM}.
%In general, these emerging technologies increase the complexity of the system and with it, the difficulty of programming it.
\section{Programming Multicores}

As already mentioned, programming is notoriously difficult since it translates from the level of abstraction of human interactions to the instructions of a computing system.
The multicore era greatly aggravates this already difficult problem.
Reasoning about concurrency, parallelism and heterogeneity on top of the correct functionality of software is significantly more difficult for a programmer.
In general, the productivity of developers cannot keep up with the pace of developments in hardware.
We speak of a ``software productivity gap''~\cite{ecker2009hardware,castrillon2014thesis}\index{software productivity gap}. 

When programming multicore systems, abstractions that proved very useful for programming single-cores become inadequate.
This is exacerbated by the problem that machines are designed for backwards-compatibility with machine models which are also sequential at the instruction level~\cite{cisnotll}.
A universal concept in programming is that of repeating an action multiple times, or looping. 
For example, consider the task of iterating through a bunch of pictures and determining which of them contain cats.
For instructing a human, we can probably say something like ``look through those pictures and sort out the ones that have cats''.
A modern x86 chip, on the other hand, would understand something closer to this:

\begin{minted}{Nasm}
LBB0_1:                                 
	cmp	dword ptr [rbp - 56], 10
	jge	LBB0_4
	mov	edi, dword ptr [rbp - 56]
	call	_read_picture
	mov	qword ptr [rbp - 64], rax
	mov	rdi, qword ptr [rbp - 64]
	call	_contains_cats
	movsxd	rcx, dword ptr [rbp - 56]
	mov	dword ptr [rbp + 4*rcx - 48], eax
	mov	eax, dword ptr [rbp - 56]
	add	eax, 1
	mov	dword ptr [rbp - 56], eax
	jmp	LBB0_1
LBB0_4:
	mov	eax, dword ptr [rbp - 52]
	mov	rcx, qword ptr [rip + ___stack_chk_guard@GOTPCREL]
	mov	rcx, qword ptr [rcx]
	mov	rdx, qword ptr [rbp - 8]
	cmp	rcx, rdx
	mov	dword ptr [rbp - 68], eax 
	jne	LBB0_6
\end{minted}

This snippet is a very oversimplified version of the task, but it serves to make the point.
Where we abstractly tell a human to look through the pictures, and they understand them as a whole set, interpreting themselves how to go through the set.
On the other hand, we instruct the machine to iterate through them by a series of very fine-grained commands.
We need to set certain registers to contain the right memory addresses, before calling an instruction to operate on them.
We then call external functions that do the reading and cat identification.
To then loop through the pictures here, simplified, we repeat this reading and identifying by jumping to a previous point in the sequence of instructions.
Even this x86 assembly snippet is already an abstraction, not only because it uses human-readable mnemonics for the instructions, but more so because it also abstracts away the concrete memory addresses and the microarchitecture.
In practice, however, almost no one would write this assembly code. Instead, they could write something closer to this (equivalent) C snippet:
\begin{minted}{C}
for(i = 0; i < N; i++){
    char *f;
    f = read_picture(i);
    results[i] = contains_cats(f); 
}
\end{minted}

Notice how the register management and several other low-level details are abstracted away.
The end of the loop is very clear to read, as we know when we have reached the final picture.
We can certainly say this is at a level of abstraction between the human and machine instructions listed above.
However, the very widespread \texttt{for} instruction we used here also has the inherently sequential semantics, as exhibited by the assembly code it translates to.
The semantics of the for loop are that the loop body will execute completely.
After each iteration of the body, the increment expression is executed (usually incrementing the iteration variable), and the condition is evaluated, deciding wether to continue iterating.
Indeed, in the two (equivalent) snippets above, we don't know how the functions \texttt{read\_picture} and \texttt{contains\_cats} work.
Do they have an inner state, or side effects?
We don't know if we can call \texttt{read\_picture} in a different order, or multiple times in parallel.
Perhaps it is internally keeping a single reference to the iterator of the image files and doing so would break the logic.
The \texttt{for} instruction is very useful to abstract away the logic of registers and instruction jumps, but not a useful abstraction for expressing concurrency.
A similar construct exists in functional programming, \texttt{map}\index{\texttt{map} function}, which generally does not have this implicit sequential semantics.
The \texttt{map} instruction is what is called a higher-order function\index{higher-order function}, taking a function as an argument and applying it to a list or any iterable object, in general.
The same cat-identifying snippet, in Haskell, can be written as follows:
\begin{minted}{Haskell}
result = map (contains_cats . read_picture) pictures
\end{minted}

While the language separates statefull and stateless computation, allowing a great analysis of concurrency, there are reasons why Haskell is not the most widespread language for embedded systems.
For example, garbage collection makes execution times very unpredictable.
Similarly, the lazyness of the language adds a performance penalty to large complex computations.
Compiling Haskell code to an efficient single-core execution is significantly more challenging than equivalent C code.
The lazyness also makes it difficult to reason about time in the computation.
This is crucial in application domains like \ac{CPS}, where the systems interact with their environment.
The \texttt{map} abstraction, as implemented in Haskell, is not well-suited for many tasks in the domain of \ac{CPS}.
In general, we are faced with trade-offs between abstract expressivity and translatability to an efficient execution.
At its core, the challenge is about choosing the right models and corresponding abstractions for a particular domain.

\section{Software Synthesis}

Models play different roles in science and engineering.
E.A. Lee explains this well in~\cite{lee2017plato}.
He argues that scientists adapt their models to fit experiments in the world, while engineers adapt designs in the world to fit their models.
Indeed, some fundamental principles of computation, like $\lambda$-calculus, are arguably discovered instead of invented, as Wadler contends~\cite{wadler2015propositions}.
Those might fit in the first paradigm, giving computer science a justification for its name.
In the case of programming multicore systems, however, the problem is clearly in the second realm: we need to engineer good models~\cite{lee2006problem}.
No serious argument can be made for languages like C or Haskell, nor the x86 instruction-set; They were invented, not discovered.

There are different ways of finding and exploiting the right models for programming multicores.
It is unlikely that there is a single right model for this.
Different models are differently suitable for different use-cases. 
For example, applicative functors in functional programming~\cite{marlow2014haxl} seem to be a great model for expressing \acsu{I/O} concurrency in microservice-based systems.
%We will discuss this example in Chapter~\ref{chap}.
As mentioned before, however, Haskell and its underlying model are not a great fit for \ac{CPS}.

For \ac{CPS} and, embedded systems in general, there is a family of methods called software synthesis~\cite{ritz1992softwaresynthesis,abbott1993softwaresynthesis,lin1998softwaresynthesis,bhartacharyya2000softwaresynthesis,pino1995softwaresynthesis,castrillon2011trends,bhattacharyya2012softwaresynthesis}\index{software synthesis}.
It is a family of methods devised precisely to help with the burden of fully exploiting the capabilities of modern multicores.
Inspired by hardware design flows, it aims to bridge the ensuing (software) productivity gap by integrating knowledge of the application and target multicore architecture into the compilation process.
At the core of these methods lies a shift in the programming model.
Instead of the de facto sequential, shared-memory model, programmers express the code in diverse \acp{MoC}.
This makes the underlying model explicit, not implicit as is the case in most programming languages.

These models expose the structure of the computation in ways that permit a compiler to reason about its parallel execution, even in the presence of heterogeneous hardware.
Aided by abstract models of the target architecture, we can design compilers for multicore systems that devise execution strategies specialized to the target architecture and applications.
Depending on the flow, the target architecture can be implicit in the methodology~\cite{ritz1992softwaresynthesis} or be an explicit input to the flow~\cite{maps}
This can be realized for example by finding efficient mappings, i.e. allocations of computational and communication resources to the different parts of an application.

As mentioned above, the central principle behind software synthesis is the underlying model of computation. 
Some approaches~\cite{lin1998softwaresynthesis} use general models, like Petri Nets~\cite{petri1962nets}, while others~\cite{ritz1992softwaresynthesis} more constrained models like \acp{SDF}~\cite{lee1987sdf}.
Most allow for multiple models~\cite{bhartacharyya2000softwaresynthesis,pino1995softwaresynthesis,bhattacharyya2012softwaresynthesis}, generally dataflow models.
Making the model explicit just makes it easier to see the trade-off between expressivity and translatability to an efficient execution.
The advantage of models like Petri nets is that they can express virtually any computation.
On the other hand, very constrained models, like \ac{SDF} provide behavioral guarantees that permit several optimizations, like static schedules and channel bounds~\cite{Parks:M95/105}.

Several more modern flows~\cite{thiele2007DOL,maps,pimentel2006systematic,kangas2006uml} have settled at the \ac{KPN} model.
Originally meant as denotational semantics for parallelism~\cite{kahn74}, the model has been shown to be compatible with dataflow~\cite{lee1995dataflow}.
Kahn Process Networks are provably deterministic~\cite{kahn74}, which is not the case for other models, e.g. Petri Nets.
In a canonical sense, \acp{KPN} are more general than most dataflow models, and represent the most general deterministic dataflow model of computation~\cite{lee_matsikoudis_semantics}.
In this thesis we will focus on a software synthesis flow~\cite{maps,castrillon2014thesis} based on \ac{KPN}.

\begin{figure}[h]
	\centering
   \resizebox{0.95\textwidth}{!}{% Define block styles
\tikzstyle{ours} = [rectangle, rounded corners, draw, fill=green!20, %legend: possible "IP leak"
    text width=7em, text centered, minimum height=4em]
\tikzstyle{other} = [rectangle, draw, fill=blue!20, 
    text width=7em, text centered, rounded corners, minimum height=4em]

\begin{tikzpicture}[node distance = 4em]
    %nodes
    \node [ours,align=center ] (moc) {application \\ (\acs*{MoC}) e.g. \acs*{KPN} };
    \node [ours, align=center, below= 0.5 cm of moc] (arch) {architecture \\ model};
    \node [ours,  align=center,below= 0.5 cm of arch] (exec) {execution \\ (e.g. traces)};
    \node (mapping) at (3.5,-0.6) [single arrow,
                                    text width=3cm,
                                    align=center,
                                    shade,shading=axis,
                                    left color=white,right color=green!20,
                                    minimum width=2.5cm,minimum height=1cm,] {\acs*{DSE} (mapping/ \\ scheduling)};
    \node (estimation) at (3.5,-3.2) [single arrow,
                                    text width=3cm,
                                    align=center,
                                    shade,shading=axis,
                                    left color=white,right color=blue!20,
                                    minimum width=2.5cm,minimum height=1cm,] {perf. estimation \\ (simulation)};

    \node [ours, minimum height=6em, align=center] (mapping) at (7.8,-0.6) {mapping \\ (\& schedule)};
    \node [other,minimum height=6em, align=center,below= 0.5cm of mapping] (simulation) {performance metrics};
    \node (generation) at (11,-1.8) [single arrow,
                                    text width=3cm,
                                    align=center,
                                    shade,shading=axis,
                                    left color=white,right color=blue!20,
                                    minimum width=2.5cm,minimum height=1cm,] {code \\ generation};

    \node [other, minimum height=6em, align=center, right = 0.5 cm of generation] (mapping)  {target \\ code};


\end{tikzpicture}
}
   \caption{A flow for \ac{MoC}-based Software Synthesis. The main abstractions colored in green are the ones we deal with in this thesis. } 
	\label{fig:software_synthesis_flows}
\end{figure}

Figure~\ref{fig:software_synthesis_flows} shows the general structure of many such \ac{MoC}-based flows. 
Application, architecture and execution models are used for performance estimation and \ac{DSE}, commonly intertwining these two processes, to produce performance metrics and a mapping (and schedule) for the application on the target architecture.
A final, code generation step generates code for the target architecture. This code is usually software for an existing architecture, but can also be hardware (e.g. in~\cite{haubelt2008systemcodesigner}), in flows that are closer to \ac{HLS}, which is the inspiration for the name software synthesis in the first place.
This thesis is limited in its scope, and it does not deal with the whole flow as depicted in Figure~\ref{fig:software_synthesis_flow}.
It deals mostly with the abstractions on the left side of the figure, colored in green.
Chapter~\ref{chap} will review these abstractions, including some of the models of computation above, and their relationships. \acp{MoC} are reviewed more generally in Chapter~\ref{chap:mocs}.

\subsection{Problems}

The methods of software synthesis are a promising avenue for programming multicores, but they still have multiple problems that make them impractical.


%Mapping structure (Part I)
Consider a concrete \ac{KPN}-based instance of the flow depicted in Figure~\ref{fig:software_synthesis_flows}, like~\cite{maps,sesame,thiele2007DOL}.
\ac{KPN} is a well-defined \ac{MoC}, used for specifying and reasoning about applications in the flow.
However, the architecture or the execution, which are comparably important, are commonly modeled in an ad hoc fashion.
With a specific architecture or topology in mind, the flows encode the architecture model with a series of performance metrics or similar numbers characterizing a specific hardware, while the structural properties are implicit in the problem formulations and algorithms.

\begin{figure}[h]
	\centering
\resizebox{0.95\textwidth}{!}{
   \begin{tikzpicture}
     \begin{scope}
%little cores
\node (pe1) at ({-1.5},{4}) [draw,minimum width=2.5cm,align=center,minimum height=2cm, fill=green!30] {\huge PE$_1$};
\node (L1-pe1) at (-1.5,2.5) [draw,minimum width=2.5cm,align=center,minimum height=1.5cm, fill=blue!30] {\huge L1\$};
\node (pe2) at (-1.5+3,4) [draw,minimum width=2.5cm,align=center,minimum height=2cm, fill=green!30] {\huge PE$_2$};
\node (L1-pe2) at (-1.5+3,2.5) [draw,minimum width=2.5cm,align=center,minimum height=1.5cm, fill=blue!30] {\huge L1\$};
\node (pe3) at (-1.5,4+4) [draw,minimum width=2.5cm,align=center,minimum height=2cm, fill=green!30] {\huge PE$_3$};
\node (L1-pe3) at (-1.5,2.5+4) [draw,minimum width=2.5cm,align=center,minimum height=1.5cm, fill=blue!30] {\huge L1\$};
\node (pe4) at (-1.5+3,4+4) [draw,minimum width=2.5cm,align=center,minimum height=2cm, fill=green!30] {\huge PE$_4$};
\node (L1-pe4) at (-1.5+3,2.5+4) [draw,minimum width=2.5cm,align=center,minimum height=1.5cm, fill=blue!30] {\huge L1\$};

%big cores
\node (pe5) at ({6.5-1.5},{4}) [draw,minimum width=2.5cm,align=center,minimum height=2cm, fill=green!60] {\huge PE$_5$};
\node (L1-pe5) at (6.5-1.5,2.5) [draw,minimum width=2.5cm,align=center,minimum height=1.5cm, fill=blue!30] {\huge L1\$};
\node (pe6) at (6.5-1.5+3,4) [draw,minimum width=2.5cm,align=center,minimum height=2cm, fill=green!60] {\huge PE$_6$};
\node (L1-pe6) at (6.5-1.5+3,2.5) [draw,minimum width=2.5cm,align=center,minimum height=1.5cm, fill=blue!30] {\huge L1\$};
\node (pe7) at (6.5-1.5,4+4) [draw,minimum width=2.5cm,align=center,minimum height=2cm, fill=green!60] {\huge PE$_7$};
\node (L1-pe7) at (6.5-1.5,2.5+4) [draw,minimum width=2.5cm,align=center,minimum height=1.5cm, fill=blue!30] {\huge L1\$};
\node (pe8) at (6.5-1.5+3,4+4) [draw,minimum width=2.5cm,align=center,minimum height=2cm, fill=green!60] {\huge PE$_8$};
\node (L1-pe8) at (6.5-1.5+3,2.5+4) [draw,minimum width=2.5cm,align=center,minimum height=1.5cm, fill=blue!30] {\huge L1\$};


\node (L2-little) at (0,0) [draw,minimum width=6cm,align=center,minimum height=1.5cm, fill=blue!55] {\huge L2\$};
\node (L2-big) at (6.5,0) [draw,minimum width=6cm,align=center,minimum height=1.5cm, fill=blue!55] {\huge L2\$};
\node (DRAM) at (3.25,-2.5) [minimum width=12.5cm,align=center,minimum height=2cm, fill=blue!85] {\huge DRAM};
%tasks
\node[ellipse,fill=red!60] (t1-1) at (pe2) {\Huge t$_1$};
\node[ellipse,fill=red!60] (t2-1) at (pe1) {\Huge t$_2$};
\draw (t1-1) edge[-{latex},line width=1.2mm, color=red!80] (t2-1);
\node[ellipse, minimum width=7cm,minimum height=4cm,dashed,ultra thick,draw] (m1) at (0,4) {};


\node[ellipse,fill=red!60] (t1-2) at (6.5-1.0+3,4.5) {\Huge t$_1$};
\node[ellipse,fill=red!60] (t2-2) at (6.5-2.0+3,3.5) {\Huge t$_2$};
\draw (t1-2) edge[-{latex},out=295,in=0,line width=1.2mm, looseness=2, color=red!80] (t2-2);
\node[ellipse, minimum width=4cm,minimum height=4cm,dashed,thick,draw] (m2) at (pe6) {};

\node[ellipse,fill=red!60] (t1-3) at (pe4) {\Huge t$_1$};
\node[ellipse,fill=red!60] (t2-3) at  (pe7){\Huge t$_2$};
\draw (t1-3) edge[-{latex},line width=1.2mm, color=red!80] (t2-3);
\node[ellipse, minimum width=8cm,minimum height=4cm,dashed,thick,draw] (m3) at (3,8) {};
\end{scope}


\begin{scope}[x=1pt,y=1pt,scale=1.5,every node/.append style={scale=1.5},xshift=400,yshift=-100]
% Created by tikzDevice version 0.12.3.1 on 2020-12-16 19:00:51
% !TEX encoding = UTF-8 Unicode
\definecolor{fillColor}{RGB}{255,255,255}
\path[use as bounding box,fill=fillColor,fill opacity=0.00] (0,0) rectangle (361.35,289.08);
\begin{scope}
\path[clip] (  0.00,  0.00) rectangle (361.35,289.08);
\definecolor{drawColor}{RGB}{255,255,255}
\definecolor{fillColor}{RGB}{255,255,255}

\path[draw=drawColor,line width= 0.6pt,line join=round,line cap=round,fill=fillColor] (  0.00,  0.00) rectangle (361.35,289.08);
\end{scope}
\begin{scope}
\path[clip] ( 36.29, 41.81) rectangle (298.86,283.58);
\definecolor{fillColor}{gray}{0.92}

\path[fill=fillColor] ( 36.29, 41.81) rectangle (298.86,283.58);
\definecolor{drawColor}{RGB}{255,255,255}

\path[draw=drawColor,line width= 0.6pt,line join=round] ( 36.29, 59.50) --
	(298.86, 59.50);

\path[draw=drawColor,line width= 0.6pt,line join=round] ( 36.29, 88.99) --
	(298.86, 88.99);

\path[draw=drawColor,line width= 0.6pt,line join=round] ( 36.29,118.47) --
	(298.86,118.47);

\path[draw=drawColor,line width= 0.6pt,line join=round] ( 36.29,147.95) --
	(298.86,147.95);

\path[draw=drawColor,line width= 0.6pt,line join=round] ( 36.29,177.44) --
	(298.86,177.44);

\path[draw=drawColor,line width= 0.6pt,line join=round] ( 36.29,206.92) --
	(298.86,206.92);

\path[draw=drawColor,line width= 0.6pt,line join=round] ( 36.29,236.41) --
	(298.86,236.41);

\path[draw=drawColor,line width= 0.6pt,line join=round] ( 36.29,265.89) --
	(298.86,265.89);

\path[draw=drawColor,line width= 0.6pt,line join=round] ( 55.51, 41.81) --
	( 55.51,283.58);

\path[draw=drawColor,line width= 0.6pt,line join=round] ( 87.53, 41.81) --
	( 87.53,283.58);

\path[draw=drawColor,line width= 0.6pt,line join=round] (119.55, 41.81) --
	(119.55,283.58);

\path[draw=drawColor,line width= 0.6pt,line join=round] (151.57, 41.81) --
	(151.57,283.58);

\path[draw=drawColor,line width= 0.6pt,line join=round] (183.59, 41.81) --
	(183.59,283.58);

\path[draw=drawColor,line width= 0.6pt,line join=round] (215.61, 41.81) --
	(215.61,283.58);

\path[draw=drawColor,line width= 0.6pt,line join=round] (247.63, 41.81) --
	(247.63,283.58);

\path[draw=drawColor,line width= 0.6pt,line join=round] (279.65, 41.81) --
	(279.65,283.58);
\definecolor{drawColor}{RGB}{0,0,0}
\definecolor{fillColor}{RGB}{68,1,84}

\path[draw=drawColor,line width= 0.1pt,line cap=rect,fill=fillColor] ( 39.50, 44.76) rectangle ( 71.52, 74.25);
\definecolor{fillColor}{RGB}{64,73,136}

\path[draw=drawColor,line width= 0.1pt,line cap=rect,fill=fillColor] ( 71.52, 44.76) rectangle (103.54, 74.25);

\path[draw=drawColor,line width= 0.1pt,line cap=rect,fill=fillColor] (103.54, 44.76) rectangle (135.56, 74.25);

\path[draw=drawColor,line width= 0.1pt,line cap=rect,fill=fillColor] (135.56, 44.76) rectangle (167.58, 74.25);
\definecolor{fillColor}{RGB}{67,59,127}

\path[draw=drawColor,line width= 0.1pt,line cap=rect,fill=fillColor] (167.58, 44.76) rectangle (199.60, 74.25);

\path[draw=drawColor,line width= 0.1pt,line cap=rect,fill=fillColor] (199.60, 44.76) rectangle (231.62, 74.25);

\path[draw=drawColor,line width= 0.1pt,line cap=rect,fill=fillColor] (231.62, 44.76) rectangle (263.64, 74.25);

\path[draw=drawColor,line width= 0.1pt,line cap=rect,fill=fillColor] (263.64, 44.76) rectangle (295.66, 74.25);
\definecolor{fillColor}{RGB}{64,73,136}

\path[draw=drawColor,line width= 0.1pt,line cap=rect,fill=fillColor] ( 39.50, 74.25) rectangle ( 71.52,103.73);
\definecolor{fillColor}{RGB}{68,1,84}

\path[draw=drawColor,line width= 0.1pt,line cap=rect,fill=fillColor] ( 71.52, 74.25) rectangle (103.54,103.73);
\definecolor{fillColor}{RGB}{64,73,136}

\path[draw=drawColor,line width= 0.1pt,line cap=rect,fill=fillColor] (103.54, 74.25) rectangle (135.56,103.73);

\path[draw=drawColor,line width= 0.1pt,line cap=rect,fill=fillColor] (135.56, 74.25) rectangle (167.58,103.73);
\definecolor{fillColor}{RGB}{67,59,127}

\path[draw=drawColor,line width= 0.1pt,line cap=rect,fill=fillColor] (167.58, 74.25) rectangle (199.60,103.73);

\path[draw=drawColor,line width= 0.1pt,line cap=rect,fill=fillColor] (199.60, 74.25) rectangle (231.62,103.73);

\path[draw=drawColor,line width= 0.1pt,line cap=rect,fill=fillColor] (231.62, 74.25) rectangle (263.64,103.73);

\path[draw=drawColor,line width= 0.1pt,line cap=rect,fill=fillColor] (263.64, 74.25) rectangle (295.66,103.73);
\definecolor{fillColor}{RGB}{64,73,136}

\path[draw=drawColor,line width= 0.1pt,line cap=rect,fill=fillColor] ( 39.50,103.73) rectangle ( 71.52,133.21);

\path[draw=drawColor,line width= 0.1pt,line cap=rect,fill=fillColor] ( 71.52,103.73) rectangle (103.54,133.21);
\definecolor{fillColor}{RGB}{68,1,84}

\path[draw=drawColor,line width= 0.1pt,line cap=rect,fill=fillColor] (103.54,103.73) rectangle (135.56,133.21);
\definecolor{fillColor}{RGB}{64,73,136}

\path[draw=drawColor,line width= 0.1pt,line cap=rect,fill=fillColor] (135.56,103.73) rectangle (167.58,133.21);
\definecolor{fillColor}{RGB}{67,59,127}

\path[draw=drawColor,line width= 0.1pt,line cap=rect,fill=fillColor] (167.58,103.73) rectangle (199.60,133.21);

\path[draw=drawColor,line width= 0.1pt,line cap=rect,fill=fillColor] (199.60,103.73) rectangle (231.62,133.21);

\path[draw=drawColor,line width= 0.1pt,line cap=rect,fill=fillColor] (231.62,103.73) rectangle (263.64,133.21);

\path[draw=drawColor,line width= 0.1pt,line cap=rect,fill=fillColor] (263.64,103.73) rectangle (295.66,133.21);
\definecolor{fillColor}{RGB}{64,73,136}

\path[draw=drawColor,line width= 0.1pt,line cap=rect,fill=fillColor] ( 39.50,133.21) rectangle ( 71.52,162.70);

\path[draw=drawColor,line width= 0.1pt,line cap=rect,fill=fillColor] ( 71.52,133.21) rectangle (103.54,162.70);

\path[draw=drawColor,line width= 0.1pt,line cap=rect,fill=fillColor] (103.54,133.21) rectangle (135.56,162.70);
\definecolor{fillColor}{RGB}{68,1,84}

\path[draw=drawColor,line width= 0.1pt,line cap=rect,fill=fillColor] (135.56,133.21) rectangle (167.58,162.70);
\definecolor{fillColor}{RGB}{67,59,127}

\path[draw=drawColor,line width= 0.1pt,line cap=rect,fill=fillColor] (167.58,133.21) rectangle (199.60,162.70);

\path[draw=drawColor,line width= 0.1pt,line cap=rect,fill=fillColor] (199.60,133.21) rectangle (231.62,162.70);

\path[draw=drawColor,line width= 0.1pt,line cap=rect,fill=fillColor] (231.62,133.21) rectangle (263.64,162.70);

\path[draw=drawColor,line width= 0.1pt,line cap=rect,fill=fillColor] (263.64,133.21) rectangle (295.66,162.70);
\definecolor{fillColor}{RGB}{167,216,71}

\path[draw=drawColor,line width= 0.1pt,line cap=rect,fill=fillColor] ( 39.50,162.70) rectangle ( 71.52,192.18);

\path[draw=drawColor,line width= 0.1pt,line cap=rect,fill=fillColor] ( 71.52,162.70) rectangle (103.54,192.18);

\path[draw=drawColor,line width= 0.1pt,line cap=rect,fill=fillColor] (103.54,162.70) rectangle (135.56,192.18);

\path[draw=drawColor,line width= 0.1pt,line cap=rect,fill=fillColor] (135.56,162.70) rectangle (167.58,192.18);
\definecolor{fillColor}{RGB}{179,219,68}

\path[draw=drawColor,line width= 0.1pt,line cap=rect,fill=fillColor] (167.58,162.70) rectangle (199.60,192.18);
\definecolor{fillColor}{RGB}{253,231,37}

\path[draw=drawColor,line width= 0.1pt,line cap=rect,fill=fillColor] (199.60,162.70) rectangle (231.62,192.18);

\path[draw=drawColor,line width= 0.1pt,line cap=rect,fill=fillColor] (231.62,162.70) rectangle (263.64,192.18);

\path[draw=drawColor,line width= 0.1pt,line cap=rect,fill=fillColor] (263.64,162.70) rectangle (295.66,192.18);
\definecolor{fillColor}{RGB}{167,216,71}

\path[draw=drawColor,line width= 0.1pt,line cap=rect,fill=fillColor] ( 39.50,192.18) rectangle ( 71.52,221.66);

\path[draw=drawColor,line width= 0.1pt,line cap=rect,fill=fillColor] ( 71.52,192.18) rectangle (103.54,221.66);

\path[draw=drawColor,line width= 0.1pt,line cap=rect,fill=fillColor] (103.54,192.18) rectangle (135.56,221.66);

\path[draw=drawColor,line width= 0.1pt,line cap=rect,fill=fillColor] (135.56,192.18) rectangle (167.58,221.66);
\definecolor{fillColor}{RGB}{253,231,37}

\path[draw=drawColor,line width= 0.1pt,line cap=rect,fill=fillColor] (167.58,192.18) rectangle (199.60,221.66);
\definecolor{fillColor}{RGB}{179,219,68}

\path[draw=drawColor,line width= 0.1pt,line cap=rect,fill=fillColor] (199.60,192.18) rectangle (231.62,221.66);
\definecolor{fillColor}{RGB}{253,231,37}

\path[draw=drawColor,line width= 0.1pt,line cap=rect,fill=fillColor] (231.62,192.18) rectangle (263.64,221.66);

\path[draw=drawColor,line width= 0.1pt,line cap=rect,fill=fillColor] (263.64,192.18) rectangle (295.66,221.66);
\definecolor{fillColor}{RGB}{167,216,71}

\path[draw=drawColor,line width= 0.1pt,line cap=rect,fill=fillColor] ( 39.50,221.66) rectangle ( 71.52,251.15);

\path[draw=drawColor,line width= 0.1pt,line cap=rect,fill=fillColor] ( 71.52,221.66) rectangle (103.54,251.15);

\path[draw=drawColor,line width= 0.1pt,line cap=rect,fill=fillColor] (103.54,221.66) rectangle (135.56,251.15);

\path[draw=drawColor,line width= 0.1pt,line cap=rect,fill=fillColor] (135.56,221.66) rectangle (167.58,251.15);
\definecolor{fillColor}{RGB}{253,231,37}

\path[draw=drawColor,line width= 0.1pt,line cap=rect,fill=fillColor] (167.58,221.66) rectangle (199.60,251.15);

\path[draw=drawColor,line width= 0.1pt,line cap=rect,fill=fillColor] (199.60,221.66) rectangle (231.62,251.15);
\definecolor{fillColor}{RGB}{179,219,68}

\path[draw=drawColor,line width= 0.1pt,line cap=rect,fill=fillColor] (231.62,221.66) rectangle (263.64,251.15);
\definecolor{fillColor}{RGB}{253,231,37}

\path[draw=drawColor,line width= 0.1pt,line cap=rect,fill=fillColor] (263.64,221.66) rectangle (295.66,251.15);
\definecolor{fillColor}{RGB}{167,216,71}

\path[draw=drawColor,line width= 0.1pt,line cap=rect,fill=fillColor] ( 39.50,251.15) rectangle ( 71.52,280.63);

\path[draw=drawColor,line width= 0.1pt,line cap=rect,fill=fillColor] ( 71.52,251.15) rectangle (103.54,280.63);

\path[draw=drawColor,line width= 0.1pt,line cap=rect,fill=fillColor] (103.54,251.15) rectangle (135.56,280.63);

\path[draw=drawColor,line width= 0.1pt,line cap=rect,fill=fillColor] (135.56,251.15) rectangle (167.58,280.63);
\definecolor{fillColor}{RGB}{253,231,37}

\path[draw=drawColor,line width= 0.1pt,line cap=rect,fill=fillColor] (167.58,251.15) rectangle (199.60,280.63);

\path[draw=drawColor,line width= 0.1pt,line cap=rect,fill=fillColor] (199.60,251.15) rectangle (231.62,280.63);

\path[draw=drawColor,line width= 0.1pt,line cap=rect,fill=fillColor] (231.62,251.15) rectangle (263.64,280.63);
\definecolor{fillColor}{RGB}{179,219,68}

\path[draw=drawColor,line width= 0.1pt,line cap=rect,fill=fillColor] (263.64,251.15) rectangle (295.66,280.63);
\end{scope}
\begin{scope}
\path[clip] (  0.00,  0.00) rectangle (361.35,289.08);
\definecolor{drawColor}{gray}{0.30}

\node[text=drawColor,anchor=base east,inner sep=0pt, outer sep=0pt, scale=  1.44] at ( 31.34, 54.54) {1};

\node[text=drawColor,anchor=base east,inner sep=0pt, outer sep=0pt, scale=  1.44] at ( 31.34, 84.03) {2};

\node[text=drawColor,anchor=base east,inner sep=0pt, outer sep=0pt, scale=  1.44] at ( 31.34,113.51) {3};

\node[text=drawColor,anchor=base east,inner sep=0pt, outer sep=0pt, scale=  1.44] at ( 31.34,143.00) {4};

\node[text=drawColor,anchor=base east,inner sep=0pt, outer sep=0pt, scale=  1.44] at ( 31.34,172.48) {5};

\node[text=drawColor,anchor=base east,inner sep=0pt, outer sep=0pt, scale=  1.44] at ( 31.34,201.96) {6};

\node[text=drawColor,anchor=base east,inner sep=0pt, outer sep=0pt, scale=  1.44] at ( 31.34,231.45) {7};

\node[text=drawColor,anchor=base east,inner sep=0pt, outer sep=0pt, scale=  1.44] at ( 31.34,260.93) {8};
\end{scope}
\begin{scope}
\path[clip] (  0.00,  0.00) rectangle (361.35,289.08);
\definecolor{drawColor}{gray}{0.20}

\path[draw=drawColor,line width= 0.6pt,line join=round] ( 33.54, 59.50) --
	( 36.29, 59.50);

\path[draw=drawColor,line width= 0.6pt,line join=round] ( 33.54, 88.99) --
	( 36.29, 88.99);

\path[draw=drawColor,line width= 0.6pt,line join=round] ( 33.54,118.47) --
	( 36.29,118.47);

\path[draw=drawColor,line width= 0.6pt,line join=round] ( 33.54,147.95) --
	( 36.29,147.95);

\path[draw=drawColor,line width= 0.6pt,line join=round] ( 33.54,177.44) --
	( 36.29,177.44);

\path[draw=drawColor,line width= 0.6pt,line join=round] ( 33.54,206.92) --
	( 36.29,206.92);

\path[draw=drawColor,line width= 0.6pt,line join=round] ( 33.54,236.41) --
	( 36.29,236.41);

\path[draw=drawColor,line width= 0.6pt,line join=round] ( 33.54,265.89) --
	( 36.29,265.89);
\end{scope}
\begin{scope}
\path[clip] (  0.00,  0.00) rectangle (361.35,289.08);
\definecolor{drawColor}{gray}{0.20}

\path[draw=drawColor,line width= 0.6pt,line join=round] ( 55.51, 39.06) --
	( 55.51, 41.81);

\path[draw=drawColor,line width= 0.6pt,line join=round] ( 87.53, 39.06) --
	( 87.53, 41.81);

\path[draw=drawColor,line width= 0.6pt,line join=round] (119.55, 39.06) --
	(119.55, 41.81);

\path[draw=drawColor,line width= 0.6pt,line join=round] (151.57, 39.06) --
	(151.57, 41.81);

\path[draw=drawColor,line width= 0.6pt,line join=round] (183.59, 39.06) --
	(183.59, 41.81);

\path[draw=drawColor,line width= 0.6pt,line join=round] (215.61, 39.06) --
	(215.61, 41.81);

\path[draw=drawColor,line width= 0.6pt,line join=round] (247.63, 39.06) --
	(247.63, 41.81);

\path[draw=drawColor,line width= 0.6pt,line join=round] (279.65, 39.06) --
	(279.65, 41.81);
\end{scope}
\begin{scope}
\path[clip] (  0.00,  0.00) rectangle (361.35,289.08);
\definecolor{drawColor}{gray}{0.30}

\node[text=drawColor,anchor=base,inner sep=0pt, outer sep=0pt, scale=  1.44] at ( 55.51, 26.95) {1};

\node[text=drawColor,anchor=base,inner sep=0pt, outer sep=0pt, scale=  1.44] at ( 87.53, 26.95) {2};

\node[text=drawColor,anchor=base,inner sep=0pt, outer sep=0pt, scale=  1.44] at (119.55, 26.95) {3};

\node[text=drawColor,anchor=base,inner sep=0pt, outer sep=0pt, scale=  1.44] at (151.57, 26.95) {4};

\node[text=drawColor,anchor=base,inner sep=0pt, outer sep=0pt, scale=  1.44] at (183.59, 26.95) {5};

\node[text=drawColor,anchor=base,inner sep=0pt, outer sep=0pt, scale=  1.44] at (215.61, 26.95) {6};

\node[text=drawColor,anchor=base,inner sep=0pt, outer sep=0pt, scale=  1.44] at (247.63, 26.95) {7};

\node[text=drawColor,anchor=base,inner sep=0pt, outer sep=0pt, scale=  1.44] at (279.65, 26.95) {8};
\end{scope}
\begin{scope}
\path[clip] (  0.00,  0.00) rectangle (361.35,289.08);
\definecolor{drawColor}{RGB}{0,0,0}

\node[text=drawColor,anchor=base,inner sep=0pt, outer sep=0pt, scale=  1.80] at (167.58,  9.00) {T$_1$ mapping (PE)};
\end{scope}
\begin{scope}
\path[clip] (  0.00,  0.00) rectangle (361.35,289.08);
\definecolor{drawColor}{RGB}{0,0,0}

\node[text=drawColor,rotate= 90.00,anchor=base,inner sep=0pt, outer sep=0pt, scale=  1.80] at ( 17.90,162.70) {T$_2$ mapping (PE)};
\end{scope}
\begin{scope}
\path[clip] (  0.00,  0.00) rectangle (361.35,289.08);
\definecolor{fillColor}{RGB}{255,255,255}

\path[fill=fillColor] (309.86,108.61) rectangle (355.85,216.78);
\end{scope}
\begin{scope}
\path[clip] (  0.00,  0.00) rectangle (361.35,289.08);
\node[inner sep=0pt,outer sep=0pt,anchor=south west,rotate=  0.00] at (315.36, 114.11) {
	\pgfimage[width= 14.45pt,height= 72.27pt,interpolate=true]{generated/2d_mapping_heatmap_ras1}};
\end{scope}
\begin{scope}
\path[clip] (  0.00,  0.00) rectangle (361.35,289.08);
\definecolor{drawColor}{RGB}{0,0,0}

\node[text=drawColor,anchor=base west,inner sep=0pt, outer sep=0pt, scale=  1.80] at (315.36,197.13) {time};
\end{scope}
\begin{scope}
\path[clip] (  0.00,  0.00) rectangle (361.35,289.08);
\definecolor{drawColor}{RGB}{255,255,255}

\path[draw=drawColor,line width= 0.2pt,line join=round] (315.36,117.22) -- (318.25,117.22);

\path[draw=drawColor,line width= 0.2pt,line join=round] (315.36,135.99) -- (318.25,135.99);

\path[draw=drawColor,line width= 0.2pt,line join=round] (315.36,154.76) -- (318.25,154.76);

\path[draw=drawColor,line width= 0.2pt,line join=round] (315.36,173.53) -- (318.25,173.53);

\path[draw=drawColor,line width= 0.2pt,line join=round] (326.92,117.22) -- (329.81,117.22);

\path[draw=drawColor,line width= 0.2pt,line join=round] (326.92,135.99) -- (329.81,135.99);

\path[draw=drawColor,line width= 0.2pt,line join=round] (326.92,154.76) -- (329.81,154.76);

\path[draw=drawColor,line width= 0.2pt,line join=round] (326.92,173.53) -- (329.81,173.53);
\end{scope}

\end{scope}


\draw (m1.south east) edge[-latex,out=295,in=150,dashed,ultra thick] (25.7,-2.2);
\draw (m2.east) edge[-latex,out=345,in=225,dashed,ultra thick] (34.1,5.6);
\draw (m3.east) edge[-latex,out=15,in=150,dashed,ultra thick] (29.1,7.1);
   \end{tikzpicture}
 }
   \caption{An example of the mapping space for a simple two-task application.}
   \label{fig:mapping_space_motivation}
\end{figure}


Consider the very simple example of mapping two processes $t_1,t_2$ where $t_2$ depends on $t_1$ onto an off-the-shelf multicore like Odroid XU4, which features a Samsung\texttrademark Exynos~5 chip with an octacore (4+4) ARM big.LITTLE\texttrademark, as depicted in Figure~\ref{fig:mapping_space_motivation}.
The mappings are plotted by encoding the mapping of each of the two tasks as the $x$ and $y$ coordinates of the grid, and the color of the squares in the grid encodes the execution time of the application with that mapping.
The example is very simple and we deliberately chose it such that it can be easily visualized.
In particular, the actual values of the execution time are irrelevant for this discussion and have been omitted.
Still, many of the problems are clear with this simple example already.

There are exactly $2^8 = 64$ mappings in the mapping space, yet only 6 distinct execution times (colors).
This is because, at least a priori, mapping both tasks to $\PE_6$, as shown in the figure, or mapping them both to $\PE_7$, will obviously result in the same execution time, since the two cores are identical (Cortex-A15\texttrademark).
This can be generally understood as a property of the \emph{symmetries} of the architecture, and should be exploited when exploring this mapping space.

Similarly, researchers often use heuristics based on geometric properties of the design space to explore it.
Yet they often disregard the encoding they use for the design space.
If we consider the point $(4,4)$ on Figure~\ref{fig:mapping_space_example}, there are four points adjacent to it, yet they are vastly different in terms of their execution time.
We can compare the geometry of the space with the geometry of the architecture itself, and see why this is the case: the distances in this space do not reflect the architecture with its heterogeneity and its memory subsystem.
In general, the geometry of this space does not reflect the geometry of the problem. 

%Execution traces directly reflect the structure of the application, as they are bound by the semantics of it.
%This is even more so the case for 
Mappings encode the resource allocation for the application to an architecture.
As such, they inherit structural properties of both the application and its semantics, as well as of the underlying architecture.
Yet mappings are commonly treated as simple lists of assignment, disregarding this structure, like which tasks depend on which, or if they are mapped to cores with a large communication latency between them.
Mapping algorithms commonly encode a heterogeneous architecture as a list of numbers of cores of different types, or perhaps use a grid system to encode \acp{PE} as they assume a \ac{NoC} with a regular-mesh topology.
These models break down as soon as the complexities of the architecture transcend the fixed model, for example by having multiple clusters or levels of hierarchy, or star-mesh topologies instead of regular meshes.

%Memory
The problems mentioned here permeate the design of the internal algorithms in software synthesis flows, which effectively constraints them to a small class of models or disregards opportunities for reasoning about the structure of the problem. 
While memory has been identified as a first-class citizen for achieving efficient implementations, many methods also consider it just as an afterthought.
For example, when discussing heterogeneity in architectures, the heterogeneity implied by the memory subsystem is seldom considered, nor are emerging memory technologies like \acp{NVM}.

%Limits of the model (KPN)
The issues raised above are not inherent issues with the flow, but rather with the state of practice.
However, the flow itself does has some inherent limitations as well.
The \ac{KPN} model of computation falls short on certain use-cases.
For example, the blocking-read semantics common in \ac{KPN} implementations are ill-suited for certain cases of data-level parallelism.
Also, perhaps more importantly, \acp{KPN} do not model time in the physical world, which plays a central role for the execution of \acp{CPS}.
In general, a model-based design approach needs to evolve its models according to the use-cases.

%Linear flow (transformations)
Another inherent problem with the flow as formulated is the structure of the flow itself.
An application is described using a concrete \ac{MoC} and then this is used to reason about an implementation.
However, the flow as depicted in Figure~\ref{fig:software_synthesis_flows} (and implemented in practice) disregards transformations at the level of the application.
This could mean a feedback loop back to the application, or perhaps semantics-preserving code transformations at the model level, as part of the exploration.

%Benchmarking
If methods like software synthesis are to be used in practice, we should also make sure they also work in practice. 
Strong results on a varied benchmark suite from real-world applications are usually a much better indicator for practical applications than, say, a good asymptotic worst-case behavior.
In order to get such results, however, we need such a varied realistic, up-to-date benchmark suite.
In reality, however, increasingly branching subdomains and concerns of intellectual property mostly yield a scarce landscape of outdated benchmarks instead.

Finally, there are multiple issues with these flows that depend more on the industry itself than the methods directly.
Tool support and maturity, degree of adoption and knowledge of the models are all beyond the realm of the academic contribution of this thesis.

\section{Contribution}

In this thesis we seek to improve the tools we use for understanding and tackling the problems discussed with software synthesis. 
We work in a model-based perspective and consider the trade-off we've introduced, between abstract expressivity and translatability to an efficient execution
To consider this we tackle the problem from both sides: the models and the compilers, in a very general sense, that translate to an efficient execution.
The main idea behind this thesis is that the underlying models endow the problem with structure.
We can then identify this structure (mathematically) and leverage it to improve our solutions.
Again there are two ways of doing this: 
\begin{enumerate}
\item by taking a concrete flow and improving it leveraging its own structure, or
\item by changing the underlying models in a way that improves the balance in some way in the trade-off above.
\end{enumerate}
This thesis discusses on both.
We first focus on software synthesis for (high-performance) embedded systems running on \acp{MPSoC}.
In particular, we focus on a concrete software synthesis flow~\cite{maps,castrillon2014thesis} based on \acp{KPN}. 
Chapter~\ref{chap:mapping} introduces this flow, as depicted in Figure~\ref{fig:software_synthesis_flows}, and the corresponding background on the mapping problem.

To evaluate methods in software synthesis in particular and compilers in general, we need to test them on benchmarks. 
Chapter~\ref{chap:benchmarking} discusses benchmarking in compilers, and introduces some benchmarks we use in the thesis.
It also discusses benchmark generation, with its advantages and pitfalls, both using random processes and machine learning.

% Exploiting Structure in Dataflow Software Synthesis: Part I
As motivated in Figure~\ref{fig:mapping_space_motivation}, the mapping problem in software synthesis has a rich structure, like its symmetries or geometry.
We identify and describe this structure in Chapter~\ref{chap:mapping_structures}.
Describing the structure is only as useful as the applications we find from it.
In Chapter~\ref{chap:mapping_applications} we discuss multiple applications, e.g. at compile time in \ac{DSE} or at run-time in hybrid mappings.
We also show how this structure can be used to formulate other properties of mappings, like robustness or compactness, which can be useful for resilient computation even in real-time scenarios in \ac{CPS}.

After exploring how to improve concrete flows with its structure, we turn our attention to the underlying models.
Chapter~\ref{chap:mocs} reviews \acfp{MoC} in general, and shows how to improve the methods here.
We first show how to improve existing methods, discussing what we call the MacQueen gap in the \ac{KPN} semantics.
We discuss a novel model, Reactors, where time is a first-class citizen, and discuss applications in the telecommunications and automotive domains.

\ac{MoC} are abstract mathematical models, they need to be exposed to programmers using a language or an \acs{API}.
In Chapter~\ref{chap:pl} we discuss the programming languages used to develop \ac{MoC}-based applications. 
We review different existing languages, including the Ohua paradigm, which can be used for implicitly defining dataflow applications.
We discuss language-level transformations and abstractions in the context of Ohua and how \ac{MoC}-based design can be used for optimizing \acs{I/O} in microservice-based architectures.

The topic of this thesis is broad, and much related work exists for all aspects covered here. 
While different chapters cover related work pertinent to the topic discussed, we review and discuss it concisely again in Chapter~\ref{chap:related}.
Finally, some conclusions from this work are summarized in Chapter~\ref{chap:conclusions}.

While all topics covered in this thesis are related by model-based design of software, not every chapter depends on everything previous.
Figure~\ref{fig:chapter_dependencies} shows the logical dependencies of the chapters, and in some cases, the sections of the chapters in this thesis.
Any path in this graph should yield a consistent exposition of the topics discussed. 
A reader only interested in some topics can readily skip chapters and sections that are not in the path to the sections that interest them.

\begin{figure}[h]
	\centering
   \resizebox{0.95\textwidth}{!}{\begin{tikzpicture}
  \node[ellipse,draw,align=center] (mapping) {Mapping KPNs \\ to MPSoCs \\ \ref{chap:mapping}};

  \node[align=center,ellipse,draw,below right = of mapping] (benchmarking) {Benchmarking \\ \ref{chap:benchmarking}};
  \node[align=center,ellipse,draw,below left = of benchmarking] (kpn_benchmarks) {KPN \\ Benchmarks \\ \ref{sec:kpn_benchmarks}};
  \node[align=center,ellipse,draw,right = of kpn_benchmarks] (level_graphs) {Random \\ Benchmarks \\ \ref{sec:level_graphs}};
  \node[align=center,ellipse,draw,below = of level_graphs] (ml) {Machine \\ Learning \\ \ref{sec:machine_learning}};
  \draw (mapping) edge[-latex] (kpn_benchmarks);
  \draw (benchmarking) edge[-latex] (kpn_benchmarks);
  \draw (benchmarking) edge[-latex] (level_graphs);
  \draw (benchmarking) edge[-latex,in=140,out=240,looseness=1] (ml);

  \node[align=center,ellipse,draw, below left = of mapping] (structures) {Structures \\ in Mappings \\ \ref{chap:mapping_structures}};
  \node[align=center,ellipse,draw,below = of structures] (metric) {Metric \\ Spaces \\ \ref{sec:metric}};
  \node[align=center,ellipse,draw,left = of metric] (symmetries) {Symmetries \\ \ref{sec:symmetries}};
  \node[align=center,ellipse,draw,below right = of symmetries] (representations) {Mapping \\ Representations \\ \ref{sec:representations}};
  \draw (mapping) edge[-latex] (structures);
  \draw (structures) edge[-latex] (symmetries);
  \draw (structures) edge[-latex] (metric);
  \draw (symmetries) edge[-latex] (representations);
  \draw (metric) edge[-latex] (representations);
  \draw (structures) edge[-latex,in=125,out=225,looseness=1] (representations);

  
  \node[align=center,ellipse,draw,below right = of representations] (compact) {Compact \\ Mappings \\ \ref{sec:compact}};
  \node[align=center,ellipse,draw,right = of compact] (design_centering) {Robust \\ Mappings \\ \ref{sec:design_centering}};
  \node[align=center,ellipse,draw,right = of representations] (logic_language) { \acs{IoT} \\ Mappings \\ \ref{sec:logic_language}};
  \node[align=center,ellipse,draw,below = of symmetries] (tetris) { Hybrid \\ Mappings \\ \ref{sec:tetris}};
  \node[align=center,ellipse,draw,below = of tetris] (dse) {\ac{DSE} \\ \ref{sec:dse}};
%  \node[align=center,ellipse,draw,below right = of tetris] (memory) {\ref{sec:memory}};
%  \draw (mapping) edge[-latex,in=45,out=225,looseness=1] (memory);
  \draw (mapping) edge[-latex,out=295,in=100] (design_centering);
  \draw (metric) edge[-latex,in=100,out=330,looseness=1] (compact);
  \draw (representations) edge[-latex] (dse);
  \draw (representations) edge[-latex] (logic_language);
  \draw (symmetries) edge[-latex] (tetris);

  \node[align=center,ellipse,draw, right = of dse] (mocs) {\acp{MoC} \\ \ref{sec:mocs_overview}};
  \node[align=center,ellipse,draw, below = of mocs] (macqueen) {The MacQueen \\ Gap \\ \ref{sec:macqueen}};
  \node[align=center,ellipse,draw, left = of macqueen] (reactors) {Reactors \\ \ref{sec:reactors}};
  \draw (mapping) edge[-latex,in=28,out=238,looseness=2] (mocs);
  \draw (mocs) edge[-latex] (macqueen);
  \draw (mocs) edge[-latex] (reactors);

  \node[align=center,ellipse,draw, right = of macqueen] (pl) {Programming \\ Languages \\ \ref{chap:pl}};
  \draw (level_graphs) edge[-latex,in=10,out=310,looseness=1] (pl);
  \draw (mocs) edge[-latex] (pl);

\end{tikzpicture}
}
   \caption{Dependencies of chapters and sections of this thesis.}
	\label{fig:chapter_dependencies}
\end{figure}


\subsection{A Note on Originality}

This thesis presents the fruits of over half a decade of research on the subjects presented.
Research, especially in an interdisciplinary approach like presented here, is much more fruitful when collaborative.
In the case of joint work, I have made an effort to focus only on my own contributions in this thesis, whenever possible.
I have also taken care to describe the work of my colleagues as theirs, when I have included it as an indispensable requirement to understand my own work.
However, some of these ideas in this thesis are the result of joint work and cannot be credited to a single person.
In those cases I have also taken care to describe the work as joint and mention other coauthors.
If in doubt, any idea or result that I have included here which has already been published elsewhere is also due to my coauthors.

%\begin{figure}[h]
%	\centering
%   \resizebox{0.55\textwidth}{!}{\begin{tikzpicture}
\draw (0,0) rectangle (5,5);
\draw (2.5,2.5) node {Placeholder};
\end{tikzpicture}
}
%	\caption{A placeholder picture.}
%	\label{fig:placeholder}
%\end{figure}
%
%\end{figure}
%\begin{figure}[h]
%	\centering
%   \resizebox{0.55\textwidth}{!}{\input{generated/placeholder_plot.tex}}
%	\caption{Placeholder plot.} %something like: https://www.karlrupp.net/wp-content/uploads/2015/06/40-years-processor-trend.png
%	\label{fig:placeholder_plot}
%\end{figure}

%A placeholder reference\cite{goens_multiprog18}.
%\begin{figure}[h]
%	\centering
%   \resizebox{0.55\textwidth}{!}{\input{figures/placeholder_flow.tex}}
%	\caption{A placeholder flow.}
%	\label{fig:placeholder_flow}
%\end{figure}

