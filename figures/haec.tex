%Based on design by Marco Miani
%https://texample.net/media/tikz/examples/TEX/swan-wave-model.tex

\begin{tikzpicture}[scale=.9,every node/.style={minimum size=1cm},on grid]

    %slanting: production of a set of n 'laminae' to be piled up. N=number of grids.
    \begin{scope}[
            yshift=-30,every node/.append style={
            yslant=0.5,xslant=-1},yslant=0.5,xslant=-1
            ]
        \fill[white,fill opacity=0.9] (0,0) rectangle (5.5,5.5);
        \draw[black,dashed] (0,0) rectangle (5.5,5.5);%marking borders
        \pic [opacity=0.5,scale=0.3125] {fourmesh};
    \end{scope}
    	
    \begin{scope}[
    	yshift=20,every node/.append style={
    	    yslant=0.5,xslant=-1},yslant=0.5,xslant=-1
    	             ]
        \fill[white,fill opacity=0.9] (0,0) rectangle (5.5,5.5);
        \draw[black,very thick] (0,0) rectangle (5.5,5.5);%marking borders
        \pic [scale=0.3125]{fourmesh};
    \end{scope}
    	
    \begin{scope}[
    	yshift=150,every node/.append style={
    	yslant=0.5,xslant=-1},yslant=0.5,xslant=-1
    	             ]
    	\fill[white,fill opacity=.9] (0,0) rectangle (5.5,5.5);
    	\draw[black,dashed] (0,0) rectangle (5.5,5.5);
        \pic [opacity=0.5,scale=0.3125] {fourmesh};
    \end{scope}
    	
    \begin{scope}[
    	yshift=200,every node/.append style={
    	    yslant=0.5,xslant=-1},yslant=0.5,xslant=-1
    	  ]
        \fill[white,fill opacity=0.6] (0,0) rectangle (5.5,5.5);
        \draw[black,dashed] (0,0) rectangle (5.5,5.5);
        \pic [opacity=0.5,scale=0.3125] {fourmesh};
      \end{scope}
    \draw[snake=zigzag] (2.8,5) -- (2.8,6.4);

    %putting arrows and labels:
    \draw[-latex,thick] (8.2,5.8) node[right]{\Large Manycore} to[out=180,in=90] (2.8,3.8);
    \draw[-latex,thick](7.8,2.3) node[right, align=center]{\Large Optical \\ \Large Interconnect} to[out=180,in=280] (3.4,3.5);
    \draw[-latex,thick](7.9,9.5) node[right,align=center]{\Large Wireless \\\Large inter-board \\\Large links} to[out=180,in=0] (3.3,6.0);
     
\end{tikzpicture}