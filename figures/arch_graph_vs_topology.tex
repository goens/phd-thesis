
\begin{scope}[xshift=-170,yshift=-80,scale=1/2,every node/.append style={scale=1/2}]
  \node[rectangle,thick,minimum width = 2cm, minimum height=2cm,fill=green!30] at (1,1) {PE$_1$};
  \node[rectangle,thick,minimum width = 2cm, minimum height=2cm,fill=green!30] at (5,1) {PE$_2$};
  \node[rectangle,thick,minimum width = 2cm, minimum height=2cm,fill=green!30] at (1,5) {PE$_3$};
  \node[rectangle,thick,minimum width = 2cm, minimum height=2cm,fill=green!30] (pe4-arch) at (5,5) {PE$_4$};
  \foreach \x in {0,...,1}{
    \foreach \y in {0,...,1}{
       \pgfmathsetmacro{\xn}{\x*4}
       \pgfmathsetmacro{\yn}{\y*4} 
       \draw ({\xn + 2}, {\yn + 2}) -- ++(0.5,0.5);
       \draw ({\xn + 2.5},{\yn + 2.5}) rectangle ++(1,1) ++(-0.5,-0.5) node[font=\Large] {R};
      }
    }

  \foreach \x in {0,...,1}{
    \foreach \y in {0,...,1}{
      \pgfmathsetmacro{\xs}{\x*4+3}
      \pgfmathsetmacro{\ys}{\y*4+3}
      \pgfmathsetmacro{\xn}{\xs+3}
      \pgfmathsetmacro{\yn}{\ys+3}
      \pgfmathsetmacro{\xsm}{\x*4+3.5}
      \pgfmathsetmacro{\ysm}{\y*4+3.5}
      \pgfmathsetmacro{\xnm}{\xsm+3}
      \pgfmathsetmacro{\ynm}{\ysm+3}
      \ifnum\x<1
      \path[<->,very thick,blue!55] (\xsm,\ys) edge (\xnm,\ys);
      \fi
      \ifnum\y<1
      \path[<->,very thick,blue!55] (\xs,\ysm) edge (\xs,\ynm);
      \fi
      }
    }

\end{scope}
\node[below = 2.4cm of pe4-arch] {$2 \times 2$ NoC Architecture};

\begin{scope}[name prefix = archgraph-]
  %cores 
  \node[ellipse,fill=green!30] (pe1) {PE$_1$};
  \node[ellipse,fill=green!30, right = of pe1] (pe2) {PE$_2$};
  \node[ellipse,fill=green!30, below = of pe1] (pe3) {PE$_3$};
  \node[ellipse,fill=green!30, right = of pe3] (pe4) {PE$_4$};

  %1 hop
  \draw[latex-latex,color=blue!55] (pe1.east) -- (pe2.west);
  \draw[latex-latex,color=blue!55] (pe1.south) -- (pe3.north);
  \draw[latex-latex,color=blue!55] (pe2.south) -- (pe4.north);
  \draw[latex-latex,color=blue!55] (pe3.east) -- (pe4.west);

  %2 hops
  \draw[latex-latex,color=blue!85] (pe1.south east) -- (pe4.north west);
  \draw[latex-latex,color=blue!85] (pe2.south west) -- (pe3.north east);

  %loops
  \draw[color=blue!30] (pe1.north) edge [loop above] ();
  \draw[color=blue!30] (pe2.north) edge [loop above] ();
  \draw[color=blue!30] (pe3.south) edge [loop below] ();
  \draw[color=blue!30] (pe4.south) edge [loop below] ();
\end{scope}
\node[below = 1 cm of archgraph-pe4] {Architecture Graph $A$};

\begin{scope}[xshift=150, name prefix = topology-]
  %cores 
  \node[ellipse,fill=green!30] (pe1) {PE$_1$};
  \node[ellipse,fill=green!30, right = of pe1] (pe2) {PE$_2$};
  \node[ellipse,fill=green!30, below = of pe1] (pe3) {PE$_3$};
  \node[ellipse,fill=green!30, right = of pe3] (pe4) {PE$_4$};

  %1 hop
  \draw[latex-latex,color=blue!55] (pe1.east) -- (pe2.west);
  \draw[latex-latex,color=blue!55] (pe1.south) -- (pe3.north);
  \draw[latex-latex,color=blue!55] (pe2.south) -- (pe4.north);
  \draw[latex-latex,color=blue!55] (pe3.east) -- (pe4.west);

  %loops
  \draw[color=blue!30] (pe1.north) edge [loop above] ();
  \draw[color=blue!30] (pe2.north) edge [loop above] ();
  \draw[color=blue!30] (pe3.south) edge [loop below] ();
  \draw[color=blue!30] (pe4.south) edge [loop below] ();

    \matrix [draw=black,fill=gray!10,right=1cm of pe2.north east, anchor=north west] {
      \node[minimum height=1em,fill=green!30] () {}; & \node[] {PE Type 1};\\
      \draw[-latex,color=blue!30] (0,-0.2) -- (0.4,-0.2); & \node[align=center] {local \\ memory};\\
      \draw[-latex,color=blue!55] (0,-0.2) -- (0.4,-0.2); & \node[] {1 hop};\\
      \draw[-latex,color=blue!85] (0,-0.2) -- (0.4,-0.2); & \node[] {2 hops};\\
    };
\end{scope}
\node[below = 1cm of topology-pe4] {Topology Graph $T$};
